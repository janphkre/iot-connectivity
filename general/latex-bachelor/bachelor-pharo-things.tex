\documentclass[12pt,a4paper]{article}
\usepackage[utf8]{inputenc}
\usepackage{graphicx}
\usepackage[a4paper, textwidth=15cm, textheight=23cm]{geometry}
\usepackage[section]{placeins}
\usepackage{listings}
\usepackage{titletoc}
\usepackage{tkz-kiviat} 
\usetikzlibrary{arrows}
\usepackage{caption}

% Kotlin template from: https://github.com/cansik/kotlin-latex-listing
\lstdefinelanguage{Kotlin}{
  comment=[l]{//},
  commentstyle={\ttfamily},
  emph={delegate, filter, first, firstOrNull, forEach, lazy, map, mapNotNull, println, return@},
  emphstyle={},
  identifierstyle={},
  keywords={abstract, actual, as, as?, break, by, class, companion, continue, data, do, dynamic, else, enum, expect, false, final, for, fun, get, if, import, in, interface, internal, is, null, object, override, package, private, public, return, set, super, suspend, this, throw, true, try, typealias, val, var, vararg, when, where, while},
  keywordstyle={\bfseries},
  morecomment=[s]{/*}{*/},
  morestring=[b]",
  morestring=[s]{"""*}{*"""},
  ndkeywords={@Deprecated, @JvmField, @JvmName, @JvmOverloads, @JvmStatic, @JvmSynthetic, Array, Byte, Double, Float, Int, Integer, Iterable, Long, Runnable, Short, String},
  ndkeywordstyle={\bfseries},
  sensitive=true,
  stringstyle={\ttfamily},
}

% Smalltalk template from:
\lstdefinelanguage{pharo}{
  commentstyle={\ttfamily},
  alsoletter={:}
  emph={},
  emphstyle={},
  identifierstyle={},
  keywords={self, super, nil, true, false, thisContext},
  keywordstyle={\bfseries},
  morecomment=[s]{"}{"},
  morestring=[b]',
  ndkeywords={new:, at:, put:, yourself, configureDynamic:, p2pServiceAdd:, p2pServiceUpdate, asString, onAnyP2P, as:, with: },
  ndkeywordstyle={\bfseries},
  sensitive=true,
  stringstyle={\ttfamily},
}

%Demotion for appendix from: https://tex.stackexchange.com/questions/61766/demoting-promoting-sections-chapters-etc
\newenvironment{leveldown}% Demote sectional commands
  {\let\section\subsection%
   \let\subsection\subsubsection%
   \let\subsubsection\paragraph%
   \let\paragraph\subparagraph%
   %\let\subparagraph\relax%
  }{}

\renewcommand{\contentsname}{Gliederung}
\renewcommand{\figurename}{Fig.}
\renewcommand{\lstlistingname}{Alg.}
\renewcommand{\lstlistlistingname}{Liste der Algorithmen}
\renewcommand{\listfigurename}{Liste der Diagramme}
\newcommand{\reflst}[1]{\lstlistingname\ \ref{#1}}
\newcommand{\reffig}[1]{\figurename\ \ref{#1}}

\title{Bachelorarbeit: PharoThings-Connectivity SS19}
\begin{document}
	\begin{titlepage}
    \includegraphics[width=0.4\textwidth]{../latex-ai-project/th_logo.png}
    ~\\[2.5cm]
    \begin{center}
    \textbf{\huge Evaluation von Bluetooth, NFC und USB im Rahmen von PharoThings und Android als p2p Verbindung}\\[0.5cm]
    {\Large Bachelorarbeit Informatik}
    \vfill
    \end{center}
    ~\\[2.0cm]
    \begin{flushright}
    {\large vorgelegt von {\bf Jan Phillip Kretzschmar}}\\[0.1cm]
    ~\\[1.5cm]
    {\large Erstprüfer: {\bf Prof. Dr. Christian Kohls}}
    ~\\[0.75cm]
    {\large Zweitprüfer: {\bf Anton Borries}}\\[0.1cm]
    {\large {\it (Zweidenker GmbH)}}\\[0.1cm]

	~\\[1.25cm]
    {\large August 2019}
	\end{flushright}
	\end{titlepage}
	\pagebreak
	\begin{titlepage}
	~\\[7.5cm]
    {\large ADRESSEN}
	~\\[1.0cm]
	\begin{flushright}
    {\large Jan Phillip Kretzschmar}\\[0.1cm]
    {\large Halzenberg 43}\\[0.1cm]
    {\large 42929 Wermelskirchen}\\[0.1cm]
    ~\\[1.0cm]
    {\large Prof. Dr. Christian Kohls}\\[0.1cm]
    {\large Campus Gummersbach}\\[0.1cm]
    {\large Steinmüllerallee 1}\\[0.1cm]
    {\large 51643 Gummersbach}\\[0.1cm]
    ~\\[1.0cm]
    {\large Anton Borries}\\[0.1cm]
    {\large Charlottenstraße 47}\\[0.1cm]
    {\large 40210 Düsseldorf}\\[0.1cm]
    ~\\[1.0cm]
    {\large Zweidenker GmbH}\\[0.1cm]
    {\large Luxemburger Straße 72}\\[0.1cm]
    {\large 50674 Köln}\\[0.1cm]
	 \end{flushright}
    \end{titlepage}
    \pagebreak
    \section*{\contentsname}
    \startcontents
    \printcontents{ }{1}{}
	 \pagebreak
	
    \section{Einleitung}
        In der Firma {\it Zweidenker GmbH} soll ein Internet of Things System auf Grundlage von PharoThings \cite{pharoThings} geschaffen werden, um Sensoren und Aktoren in einem Netzwerk oder zu einem Server verfügbar zu machen.
        Im Internet of Things (IoT) werden Geräte miteinander vernetzt, die nicht zwingend eine Internetanbindung benötigen würden, jedoch erweiterte Funktionalitäten der Geräte dem Nutzer so zur Verfügung gestellt werden. Um solchen Geräten eine Internetanbindung geben zu können, ist es nötig, eine kabelgebundene Verbindung zu schaffen oder die Geräte in ein kabelloses Wi-Fi Netzwerk aufzunehmen. Die Verbindungskonfiguration von IoT Geräten für Wi-Fi Netzwerke wurde bereits auf Basis von Wi-Fi Direct untersucht \cite{aiProject}. Da die Implementierung dieser Technologie einige Unzulänglichkeiten besitzt, soll nun ein Vergleich von weiteren p2p Technologien \linebreak vorgenommen werden. Hierunter fallen Bluetooth, Near Field Communication (NFC) und Universal Serial Bus (USB), da diese alle von Android Smartphones nativ unterstützt werden. Da die Vor- und Nachteile sowie Funktionsweisen dieser p2p Schnittstellen bereits bekannt sind, sollen sie im Rahmen dieser Arbeit in die \linebreak Verbindungskonfiguration mit PharoThings eingebunden werden, um auch hier Probleme aufzudecken und zu einer optimalen Lösung des Problems zu gelangen.
        
        Die existierende Lösung besteht aus einer Service Discovery und einer p2p \linebreak Verbindung auf Basis von Wi-Fi Direct. Die Service Discovery wird als bestehend beibehalten, um die Evaluation auf die Nutzbarkeit der p2p Verbindung zu konzentrieren. Ziel soll sein, eine Alternative p2p Technologie zu Wi-Fi Direct zu finden.

        \subsection{Zielsetzung}
        Im vorausgegangenen Praxisprojekt wurde deutlich, dass Wi-Fi Direct einige \linebreak gravierende Mängel im Hinblick auf die Funktionalität und Robustheit der bereitgestellten Bibliotheken aufweist. So ist die Dokumentation der meisten Nachrichten und Events unvollständig oder fehlt und es ist unklar wie eine Verbindung aufgebaut werden muss. Gleichzeitig sind Fehlermeldungen entweder nicht aussagekräftig oder führen dazu, dass der laufende Daemon abstürzt.
        Daraus lassen sich für diese Evaluation zwei Kriterien im Bereich der Robustheit ableiten.
        Zunächst kann getestet werden wie aussagekräftig Fehlercodes oder Fehlermeldungen sind.
        Das eine Ende des Spektrums bildet hierbei die Darstellung des Erfolgs eines Aufrufs durch eine boolesche Variable, auf der anderen Seite wird ein genauer Fehlercode in Verbindung mit einer Fehlermeldung ausgegeben. Die Fehlercodes sind dabei für jede genutzte Methode dokumentiert und bieten Aufschluss auf die Gründe der erhaltenen Fehlschläge. 
        Gleichzeitig soll evaluiert werden, in wie weit sich von aufgetretenen Fehlern erholt werden kann, um dennoch eine erfolgreiche Verbindung aufbauen zu können. Hierbei wird der messbare Bereich auf der einen Seite durch den Absturz des verwendeten Moduls beschränkt. Es besteht somit keine Möglichkeit zur Erholung von Fehlern und es muss darüber hinaus sogar das verwendete Modul neu gestartet werden. Dem gegenüber steht die gute Dokumentation von Fehlercodes, welche es erlaubt, auf nicht kritische Fehler reagieren zu können. Ebenso können die Fehler bereits einen Codeblock beinhalten, der es erlaubt, den Fehler, so fern dieser unerwünscht gewesen ist, zu beseitigen.
        Ebenso sollte die Anbindung einer eingesetzten Technologie testbar sein. Die Testbarkeit der Software ergibt sich zum einen aus einem hohen Abstraktionsgrad, so dass sich die genutzten Technologien mit wenig Aufwand durch Mocks und Stubs auswechseln lassen. Ebenso ist eine Testbarkeit erst dann gegeben, wenn die genutzt Technologie eine hohe Stabilität im Hinblick auf Wiederholbarkeit bietet.
        
        Die Funktionalität von Wi-Fi Direct wird aktuell durch mehrere Faktoren \linebreak eingeschränkt. Zunächst mangelt es an einem internen Zustandsdiagramm des genutzten Moduls. Dadurch ist es nicht möglich, festzustellen, welche Methoden in einer bestimmten Reihenfolge aufgerufen werden müssen, um den erwünschten Zustand einer Verbindung zu erreichen. Zudem sind hierbei interne Zustandsübergänge durch ankommende Events oder weitere unbekannte Gründe nicht dokumentiert. Dies führt dazu, dass nicht klar ist, wann Methoden aufgerufen werden müssen um den aktuellen Zustand der p2p Schnittstelle beizubehalten.
        Als Evaluationskriterium kann hierbei wieder die Dokumentation im Hinblick auf die Funktionalität dienen, denn es kann festgestellt werden, wie gut der Verbindungszustand und Modulzustand dokumentiert sind. Es besteht zum Einen die Möglichkeit, dass keine Dokumentation vorliegt und das genutzte Modul lediglich als Blackbox genutzt werden kann. Als Optimum sollte jedoch die Dokumentation soweit vorhanden sein, dass interne sowie externe Events dokumentiert sind und gemeinsam mit einem Zustandsdiagramm sowohl der Verbindung als auch der Software dazu genutzt werden können, genau nachzuvollziehen, wann sich der aktuelle Zustand ändert und wie dieser wiederhergestellt werden kann.
        
        Die Qualität der eingesetzten Technologie kann zudem auf Effizienz überprüft werden. Da die genutzten Module in das Gerät fest integriert sind, lässt sich Energieverbrauch nicht messen und ist außerdem von dem im Gerät konkret verbauten Chip und dessen Treibern abhängig. Es kann jedoch eine Aussage über Übertragungsraten getroffen werden, welche eine numerische Skala abbilden. Hierdurch kann festgestellt werden, in wie weit manche Anwendungsfälle mit der entsprechenden Technologie möglich sind, oder die Nutzbarkeit dadurch eingeschränkt werden. Die Konfiguration über eine REST-Schnittstelle benötigt keine hohen Datenraten zur Kommunikation, falls der Nutzer jedoch beispielsweise eine Remote-Verbindung durch Telepharo über die p2p Schnittstelle zwischen seinem persönlichen Computer und dem eingesetzten IoT-Gerät aufbauen willen, benötigt er eine relativ latenzfreie und hochvolumige Datenübertragung.
        
        Letztlich kann überprüft werden, wie leicht sich die Technologie implementieren lässt, sodass sie für das bestehende Projekt genutzt werden kann. Da die Schwierigkeit von Aufgaben im Rahmen von Softwareentwicklung immer eine Kombination aus Zeitaufwand und Komplexität sind, lässt sich hierbei immer nur eine Schätzung oder persönliche Meinung auf Basis der eigenen Präferenz angeben.
    \subsection{Evaluationsmethodik}
    	Um die genannten Evaluationsziele überprüfen zu können, soll jede der Technologien Bluetooth, NFC und USB in einem vergleichbaren Maße zu Wi-Fi Direct im Hinblick auf die Vollständigkeit ihrer Dokumentation und Nutzbarkeit ihrer Implementierung überprüft werden.
    	Unter dem Aspekt der Softwarequalität lassen sich die Technologien gegenüberstellen. Dabei soll auf Grundlage einer beispielhaften Implementierung im Rahmen des bestehenden Projektes die Vor- und Nachteile verdeutlicht werden und zudem eine Evaluation der Komplexität und des Aufwandes einer Implementierung vorgenommen werden. Metriken, um die Qualität der Software zu messen, sind als ISO-Standard 9126 definiert \cite[S.6]{liggesmeyer}. Hierbei soll auf die folgenden Punkte eingegangen werden und in wie weit diese extern gemessen werden können, da nicht einsehbar ist, wie ausgelieferte Implementierungsartefakte intern getestet werden und dokumentiert sind. Es soll so auf die Punkte Zuverlässigkeit, Wartbarkeit (Instandhaltbarkeit), Funktionalität, Effizienz und Benutzbarkeit als Softwarequalitätsmerkmale eingegangen werden.
    	\begin{enumerate}
    	\item {\it Zuverlässigkeit:} 
    	Um die Robustheit und Zuverlässigkeit der Technologien feststellen und vergleichen zu können, muss festgestellt werden können, wie viele Fehler in der genutzten Implementierung existieren \cite[S.9]{liggesmeyer}. Da dieser Punkt jedoch unter Anderem stark von der genutzten Hardware abhängt soll sich zunächst auf das Android Gerät Samsung Galaxy S7 und einen Raspberry Pi 3 B+ beschränkt werden. Die Anzahl von Fehlern in einer Technologie lässt sich ebenfalls nicht messbar festsetzen, da zwar die Zahl von gemeldeten Fehlern in einer Anwendung über öffentliche Bugtracker der eingesetzten Bibliothek nachvollziehbar ist, jedoch so nicht alle Fehler abgefangen werden können, da zum einen auch Fehler in Treibern und Hardwaredesign bestehen können und ebenso ein Bugtracker nicht zwingend die vollständige Menge aller Fehler enthalten kann. Gleichzeitig sollte hierbei zur Evaluation die Qualität der Dokumentation bewertet werden. Dazu zählt die Dokumentation und Verfügbarkeit des Sourcecodes im Hinblick auf die Schicht, welche als Schnittstelle bereit gestellt wird.
    	\item {\it Wartbarkeit:} Um Fehler bei der Benutzung einer der Technologien beheben zu können, ist es nötig, dass die Technologie auf der einen Seite durch Dokumentation und Quellcode leicht zu verstehen ist und gleichzeitig es erlaubt, reproduzierbar Fehler zu testen. Um hierbei eine aussagekräftiges Bewertung vornehmen zu können, sollte keine Metrik wie die Häufigkeit von abweichenden Ergebnissen bei gleichen Eingabeparametern genutzt werden, da dies von nicht überschaubaren Faktoren abhängt. Es kann daher nur auf den Abstraktionsgrad der angebotenen Schnittstellen eingegangen werden, da diese für eigene Tests ersetzt werden müssen. Jedoch kann dieser Punkt auch in Verbindung mit der Zuverlässigkeit gesehen werden, da Fehler bei einer sauberen Dokumentation kein großes Hindernis für Wiederholbarkeit und damit Testbarkeit darstellen.
    	\item {\it Funktionalität:}
    	Die Abdeckung der Implementierungen nach der Definition einer Vollständigkeit der erwarteten Funktionalität \cite[S.8]{liggesmeyer} kann nur sehr eingeschränkt überprüft werden, da aus einer externen Sicht die Menge der Funktionalitäten sich auf den Verbindungsaufbau und Verbindungsabbau, sowie ein Senden und Empfangen von verbindungslosen Daten beschränkt und alle Technologien diese Funktionalitäten anbieten. Es muss somit darauf geachtet werden in wie weit die internen Zustände dokumentiert sind und erreicht werden können.
    	\item {\it Effizienz:}
    	Die Effizienz der Technologien kann durch eine Vergleich von Datendurchsatzraten und Antwortzeiten festgestellt werden. Diese beiden Kennzahlen sind für eine Verbindungskonfiguration nicht kritisch, jedoch können mittels eines externen HTTP-Wrappers auch Drittprogramme eine solche p2p Verbindung nutzen und in Anwendungsfällen wie Streaming von Audio und Video werden solche Raten dann interessant. Eine andere Seite der Effizienz im Hinblick auf Energieverbrauch der Technologien wird vor Allem im Betrieb eines Gerätes mit einer Batterie wichtig. Ein Stromverbrauch lässt sich jedoch ebenfalls kaum bewerten, da dies stark von der verbauten Hardware abhängt. Da Dies auch für eine Verbindungskonfiguration zunächst unerheblich ist, wird auf eine Bewertung der Energieaufnahme verzichtet.
    	\item {\it Benutzbarkeit:}
    	Die Benutzbarkeit der eingesetzten p2p Lösung, ergibt sich zum Einen daraus, wie stark der Nutzer in die verwendete Technologie eingebunden werden muss und zum Anderen daraus, wie komplex und zeitaufwändig eine Anbindung der Technologie ist.
    	Der Nutzer ist zum größten Teil dazu angehalten, einen Verbindungsaufbau als erwünscht zu bestätigen. Dieser Vorgang sollte so leicht wie möglich gestaltet sein und kann durch eine \linebreak Gegenüberstellung der einzelnen Schritte, die der Nutzer ausführen muss, bis eine erfolgreiche Verbindung besteht, verglichen werden.
    	Die Komplexität und der Aufwand einer Implementierung kann anhand der Menge an Methoden, die implementiert werden müssen, besonders in einer C-Bibliothek, die eine eigene \linebreak Schnittstelle zu pharo abbildet, festgestellt werden. Außerdem ist ein Vergleich im Hinblick auf die Komplexität der Logik, die bei der Anbindung gehandhabt werden muss, nötig.
    	\end{enumerate}
	    	\pagebreak
    		\subsection{Messbarkeit}
		Da Softwarequalität ebenso wie die benannten Metriken größtenteils nicht quantitative messbare Kriterien sind, können auch keine Referenzwerte für die einzelnen Kriterien zu Hilfe gezogen werden. Eine quantitative Aussage wie eine Anzahl an existierenden Tests ist ebenfalls nur bedingt aussagekräftig, Da nicht bewertet werden kann, wie robust ein Test geschrieben wurde und welche Ergebnisse getestet werden. Der letztere Punkt lässt sich teilweise über eine Analyse der Quellcodeabdeckung der Tests erreichen, jedoch entspricht eine Implementierung einer Technologie nicht zwingend ihrer Spezifikation und erlaubt so dennoch Abweichungen in den getesteten Ergebnissen und somit auch Inkompatibilitäten zu anderen Implementierungen der selben Technologie.
		
		Eine Bewertung kann somit primär nur auf Grundlage der gesammelten Erfahrungswerte auf Grundlage der vorgenommenen Umsetzung mit den zu \linebreak überprüfenden Technologien getroffen werden. 
		
		\subsection{Einschränkungen}
		Um den Aufwand der Arbeit bewältigbar zu halten, gibt es ein paar weitere Rahmenbedingungen, die im Vorfeld definiert werden sollen. Hierbei ist darauf zu achten, wenn bei einer der Technologien ein deutlich größerer Aufwand abzusehen sein, sollte die Implementierung dieser Technologie ausgeklammert werden, da es auch einer Nutzbarkeit widerspricht, eine solche Technologie trotz einem hohen Aufwand in der Implementierung zu nutzen, da dies gleichzeitig auch einen hohen Aufwand in der Wartung bedeutet.
		
		Ein Bezug zwischen Kosten und Nutzen lässt sich über ein Qualitätsmodell ebenfalls nur schlecht aufstellen, da alle Technologien frei zugängliche Software besitzen und Kostenunterschiede lediglich durch die genutzte Hardware entstehen. Lediglich bei NFC entstehen Mehrkosten, da es nicht bereits im Raspberry Pi integriert ist. Es wird somit darauf verzichtet, die Nutzbarkeit der einzelnen Technologien anhand ihrer Kosten gegenüberzustellen.
		
		Weiterhin kann nicht auf Dokumentationen der entsprechenden Standards und Implementierungen der Technologien eingegangen werden, die nicht öffentlich \linebreak zugänglich sind, da ebenso Änderungen und Erweiterungen an diesen Spezifikationen dann nicht nachvollziehbar sein werden und ein Implementierungsaufwand einer eigenen Anbindung somit unvergleichbar hoch ist.
		
		

    \section{Warum p2p?}
    Das vorausgegangen Projekt einer Verbindungskonfiguration von IoT-Geräten \cite{aiProject} hat gezeigt, dass eine initiale p2p Verbindung aufgebaut werden muss, um eine Wi-Fi Verbindung konfigurieren zu können. Eine direkte Verbindung zwischen zwei Endgeräten ist nötig, wenn Daten zwischen Diesen ausgetauscht werden sollen und keine Verbindung über ein gemeinsames Netzwerk oder einen Server stattfinden kann oder soll.
    - Warum p2p

    - Präzisierung IoT - Internet of Things
    \subsection{Services}
    - Definition eines Services
    
    - Anbieten eines Services als Client/Server-Modell
    
    \subsection{Servicestabilität (Todo: entfällt eventuell)}
    - API Versionierung
    
    - Absichern von Änderungen durch Contract-Driven-Development
    
    - Integrationstests
    \subsection{Netzwerktypen}
		- LAN, WAN, etc.
		
		- ad hoc, Stern-basiert, Bus-basiert

    \section{Grundlagen der Verbindungstechnologien}
	Da die Nutzbarkeit der Übertragungstechnologien für die Verbindungskonfiguration bereits in \cite{aiProject} überprüft wurden, soll noch einmal jede Technologie kurz vorgestellt werden.
	\subsection{Bluetooth}
	
	\subsection{NFC}
	
	\subsection{USB}

    \section{Eigene Umsetzung}
        Im Folgenden wird die Implementierung einer p2p Verbindung über Bluetooth, NFC und USB beschrieben. Jede dieser Technologien soll zunächst separat betrachtet werden, um Details und Probleme bei der Implementierung aufzuzeigen.
        
        Bei der Kapselung von HTTP Anfragen muss bedacht werden, wie eine Technologie Verbindungen zur Verfügung stellt, da so zwischen kurzlebigen Verbindungen ähnlich zu HTTP Anfragen und langlebigen Verbindungen wie einer Datenübertragung per Kabel unterschieden werden muss. Letztere weisen dabei das Problem auf, HTTP Anfragen, die mit einem {\it EOF} beendet werden, über eine langlebige Verbindung zu senden, da ein {\it EOF} immer auch das Ende der Verbindung aufzeigt.
        
    \subsection{HTTP Kapselung}
        Um eine p2p Verbindung unabhängig von der genutzten Verbindungstechnologie nutzen zu können, ist es nötig, eine HTTP Verbindung zum existierenden REST-Server aufbauen zu können. Da das REST-Prinzip eng mit HTTP verbunden ist, sind auch die Implementierungen von REST meistens fest mit einem HTTP Server oder Client verbunden. Auf der Serverseite stellt dies kein Problem dar, da der HTTP-Server nicht von den genutzten Sockets entkoppelt werden muss. Dazu wird ein zweiter Server vorgeschaltet, welcher lokale Sockets nutzt um mit dem eigentlichen Server zu kommunizieren und den Verbindungsaufbau sowie Verbindungsabbau für die genutzte Technologie und deren mögliche virtuelle Socket-Verbindung zu verwalten.
        
        \begin{lstlisting}[frame=bt, label={lst:socket:create}, language=C, caption=Instanziierung eines Sockets (Servercode in C)]
int s = socket(AF_INET, SOCK_STREAM , 0);
struct sockaddr_in rem_addr = { 0 };
rem_addr.sin_addr.s_addr = htonl(INADDR_LOOPBACK);
rem_addr.sin_family = AF_INET;
rem_addr.sin_port = htons(targetPort);
connect(s, (struct sockaddr *)&rem_addr , sizeof(rem_addr));
        \end{lstlisting}
        Wie im \reflst{lst:socket:create} zu sehen, baut der kapselnde Server eine Socket-Verbindung auf, um Daten zum Zielserver zu senden. Diese Verbindung wird erst aufgebaut, sobald das Endnutzergerät kurz davor steht eine Anfrage an den Zielserver zu stellen, um mögliche Zeitüberschreitungen im HTTP-Server zu vermeiden. Dies erfüllt sonst die Anfangskriterien eines slow-louis-Angriffes auf einer einzelnen Verbindung. \footnote{Quelle?}
        \begin{lstlisting}[frame=bt, label={lst:socket:data}, language=C, caption=Datenweiterleitung durch Sockets (Servercode in C)] 
int pipeData(int sourceSocket, int sinkSocket, char* buffer) {
    int bytes_read = read(sourceSocket, buffer, BUF_SIZE);
    if(bytes_read < 0) return -1;
    if(bytes_read == 0) return -20; // indicates an EOF
    int bytes_sent = send(sinkSocket, buffer, bytes_read, 0);
    if(bytes_sent < 0) return -2;
    return 0;
}
        \end{lstlisting}
        
        Die bestehende Socket-Verbindung zum Zielserver wird genutzt, um die ankommenden Daten und deren Antworten voll duplex weiterzuleiten. Dies ist realisiert, indem eine Methode {\it pipeData} in zwei Threads mit invertiertnm {\it sourceSocket} und {\it sinkSocket} aufgerufen wird, bis ein {\it EOF} gesendet wird (\reflst{lst:socket:data}). Für den kapselnden Server ist es so unerheblich, welche Seite Daten zuerst senden möchte. Die Verbindung zwischen dem Zielserver und dem kapselnden Server kann so auch unabhängig von der Verbindung zum Endgerät verwaltet werden, da beim Lesen der Daten vom Zielserver ein{\it EOF} das Ende der Verbindung zum Zielserver aufzeigt. Die Verbindung zum Endgerät muss so nicht zwingend geschlossen werden.
        
        Ein Abkapseln der Technologie über einen weiteren Server auch auf Seite des Clients erscheint als vermeidbarer Overhead, da so keine Serverimplementierung auch im Client stattfinden muss. Stattdessen wird hier der HTTP-Client, welcher der REST-Bibliothek zugrunde liegt, so aufgetrennt, dass keine TCP/IP-Verbindungen aufgebaut werden, jedoch die Anfragen aus der Bibliothek als String entnommen werden können und  deren Antworten als String eingespeist werden können. Für den REST-Client retrofit wird intern okhttp\footnote{Referenz} als HTTP-Client genutzt. Um zu verstehen, welche Änderungen am Client nötig sind, sollte zuerst die interne Struktur der okhttp Bibliothek erläutert werden.
        \begin{lstlisting}[frame=bt, label={lst:android:okhttp}, language=Java, caption=Interner Aufbau von okhttp (Clientcode in Java)]
List<Interceptor> interceptors = new ArrayList<>();
...
interceptors.add(retryAndFollowUpInterceptor);
interceptors.add(new BridgeInterceptor(client.cookieJar()));
interceptors.add(new CacheInterceptor(client.internalCache()));
interceptors.add(new ConnectInterceptor(client));
...
interceptors.add(new CallServerInterceptor(forWebSocket));
        \end{lstlisting}
        Serveraufrufe werden im okhttp Client (\reflst{lst:android:okhttp})\footnote{\cite{okhttpRealCall} Zeilen 185-194)} durch eine Kette von Interceptors verarbeitet. Jeder {\it Interceptor} hat dabei die Möglichkeit den Aufruf oder die Kette beliebig zu verändern und in dieser Liste nimmt jeder Interceptor eine andere Rolle ein. Der {\it RetryAndFollowUpInterceptor} stellt in der Kette eine {\it StreamAllocation} bereit und übernimmt das Abbrechen von Aufrufen. Der darauf folgende {\it BridgeInterceptor} verwaltet Cookies aus den Anfragen und Antworten, sowie die Übersetzung von Anwendungsanfragen zu Netzwerkanfragen. Dies beinhaltet ebenfalls die Verwaltung von netzwerkrelevanten Headern wie zum Beispiel den "User-Agent"-Header oder "Content-Encoding"-Header. Wie der Name des {\it CacheInterceptor} bereits vermuten lässt, wird in diesem das Speichern und Abrufen von Antworten auf wiederkehrende Anfragen ermöglicht. Sowohl Cookies als auch der Cache lassen sich einfach umgehen, indem der Client jeweils kein Objekt ausliefert oder ein Objekt bereitstellt, welche alle Methoden mit leeren Ergebnissen quittiert. Bevor der Aufruf vom {\it CallServerInterceptor} tatsächlich ausgeführt wird und auf ein Ergebnis gewartet wird, Baut der {\it ConnectInterceptor} noch eine HTTP-Verbindung über die {\it StreamAllocation} der Kette auf. Für diese Verbindung wird dann ein {\it HTTPCodec} genutzt, um die Anfrage auf den Socket zu schreiben und die Antwort zu lesen.
        
        \begin{lstlisting}[frame=bt, label={lst:android:okhttpchanges}, language=Java, caption=Änderungen an okhttp (Clientcode in Kotlin)]
val interceptors = ArrayList<Interceptor>()
...
interceptors.add(BridgeInterceptor(wrapper.cookieJar()))
interceptors.add(CacheInterceptor(wrapper.internalCache()))
...
interceptors.add(SimpleServerInterceptor(wrapper.httpCodec()))
        \end{lstlisting}
        \footnote{Todo: Rahmen um Code} Im Gegensatz zur okhttp Implementierung muss die p2p Verbindung so verwaltet werden, wie es von der Implementierung der Technologie vorgegeben wird. Dazu wird der {\it RetryAndFollowUpInterceptor} sowie der {\it ConnectInterceptor} weggelassen und der {\it CallServerInterceptor} im {\it SimpleServerInterceptor} soweit vereinfacht wird, sodass dieser keine Handshakes mehr unterstützt und nicht mit Websockets genutzt werden kann.
        Um die Kapselung so simpel wie möglich zu halten, wird ebenfalls auf HTTP 2 verzichtet, wodurch der abgewandelte HTTP Client nur HTTP1.1 unterstützt. Da im okhttp Client der {\it HttpCodec} sowohl die Aufgabe erfüllt, den Request in einen HTTP-String umzuwandeln, als auch den Request über die Verbindung zu schreiben, muss so lediglich eine weitere Klasse angepasst werden. der {\it Http1Codec} wird dabei minimal angepasst, sodass interne Klassen der okhttp Implementierung, die nicht im Rahmen dieser HTTP Kapselung nötig sind, genutzt werden.
        
        
        Diese generische HTTP Kapselung lässt sich nun ähnlich der okhttp Implementierung über eine zentrale Klasse, den {\it SimpleHttpWrapper} nutzen. Diese Klasse nutzt ebenfalls das Builder-Pattern, um so nah wie möglich an der okhttp Bilbiothek zu bleiben. Über diesen Builder lässt sich nun ein {\it ConnectionStream} definieren, welcher dann den Inputstream und Outputstream der Verbindung bereitstellt.
        
        \subsection{Bluetooth}
        Die Umsetzung einer p2p Verbindung über Bluetooth besteht darin, dass ähnlich zu HTTP Sockets, RFCOMM Sockets genutzt werden, um mit einem Server kurzweilig zu kommunizieren. Jede dieser RFCOMM Verbindungen bildet hierbei eine HTTP Anfrage und HTTP Antwort ab. Wie bereits in der HTTP Kapselung beschrieben, wird die Verbindung von selbst wieder geschlossen, sobald ein EOF gesendet wird. Da RFCOMM Socket Verbindungen ein automatisches Pairing mit Schlüsselaustausch durchführen, ist keinerlei Eingriff oder Bestätigung des Nutzers nötig, um Daten übertragen zu können.
        \begin{lstlisting}[frame=bt, label={lst:android:bluetooth}, language=Java, caption=Verbindungsaufbau mit Bluetooth (Clientcode in Kotlin)]
val bluetoothDevice = bluetoothAdapter.bondedDevices?.firstOrNull { bluetoothDevice ->
    bluetoothDevice.address == device.bluetoothDetails.mac
} ?: bluetoothAdapter.getRemoteDevice(device.bluetoothDetails.mac)

val method = bluetoothDevice::class.java.getMethod("createInsecureRfcommSocket", Int::class.javaPrimitiveType)
val socket = method.invoke(bluetoothDevice, device.bluetoothDetails.port) as BluetoothSocket
socket.connect()
        \end{lstlisting}
        Für den Verbindungsaufbau muss die Bluetooth MAC-Addresse sowie der Bluetooth Port angegeben werden. Der {\it BluetoothAdapter} aus dem Android Framework gibt ein {\it BluetoothDevice} für die angegebene MAC-Addresse zurück. Es ist hierbei noch unerheblich, dass dieses Gerät auch in der Nähe erreichbar ist oder existiert. Über die versteckten Methoden {\it BluetoothDevice::createInsecureRfcommSocket} und {\it BluetoothDevice::createRfcommSocket} kann eine Socket-Verbindung zu einem bestimmten Port des entfernten Gerätes erstellt werden. In Android sind diese Methoden versteckt, um Konflikte zwischen Apps beim festlegen der Portnummern zu vermeiden, da lediglich 30 RFCOMM Ports zur Verfügung stehen.\footnote{Quelle z.B. https://people.csail.mit.edu/albert/bluez-intro/x148.html} Stattdessen sollen Services eine UUID generieren, welche über Bluetooth Service Discovery Protocol (SDP) von anderen Geräten abgefragt werden kann. Das SDP Protokoll übernimmt dann die Vergabe von Ports für die registrierten Services. Um diese exemplarische Implementierung jedoch simpel zu halten und volle Kontrolle über das Servergerät besteht, wird hierbei ein vordefinierter Port genutzt.

         Jegliche Logik zum tatsächlichen Verbindungsaufbau ist in der Methode {\it BluetoothSocket::connect} gekapselt und muss bei der Umsetzung nicht beachtet werden. Auf beiden Geräten muss lediglich Bluetooth eingeschaltet sein und das Servergerät muss auffindbar für den Client sein.
   
        \begin{lstlisting}[frame=bt, label={lst:bluetooth:socket}, language=C, caption=Verbindungsaufbau mit Bluetooth (Servercode in C)]
int s = socket(AF_BLUETOOTH, SOCK_STREAM, BTPROTO_RFCOMM);
struct sockaddr_rc loc_addr = { 0 };
loc_addr.rc_family = AF_BLUETOOTH;
loc_addr.rc_bdaddr = *BDADDR_ANY;
loc_addr.rc_channel = (uint8_t) bluetoothPort;
bind(s, (struct sockaddr *)&loc_addr, sizeof(loc_addr));
listen(s, LISTEN_QUEUE_SIZE);
        \end{lstlisting}
        Auf Seite des Servers kann die Implementierung der HTTP Kapselung fast vollständig übernommen werden,jedoch müssen Verbindungen auf einem weiteren Socket mit dem Bluetooth RFCOMM Protokoll akzeptiert werden.
        
        Mit diesen beiden Anbindungen von Bluetooth an die generische HTTP Kapselung lässt sich diese simple Lösung bereits nutzen. Um jedoch die geheimen Methoden unter Android nicht nutzen zu müssen, ist es nötig, auf dem Server die Anbindung im Bluetooth SDP als Service zu hinterlegen. Kürzlich wurde die BlueZ 5 Bibliothek von einer simplen C-Schnittstelle auf eine DBus-Schnittstelle umgewandelt.\footnote{\cite{BluezMigration}} Dies hat zur Folge, dass viele der Beispiele und Erklärungen, ebenso wie Bücher nicht mehr aktuell sind und erst auf die neue API hingewiesen wird, wenn nach der expliziten Fehlermeldung gesucht wird.
        
        \begin{lstlisting}[frame=bt, label={lst:bluetooth:sdp}, language=C, caption=Veraltete Nutzung von SDP (Servercode in C)]
sdp_set_info_attr(record, serviceName, serviceProvider, serviceDescription);
*session = sdp_connect(BDADDR_ANY, BDADDR_LOCAL, SDP_RETRY_IF_BUSY);
sdp_record_register(*session, record, 0);
        \end{lstlisting}
        In älteren Versionen von BlueZ war es möglich einen SDP Eintrag mit der Methode {\it sdp\_record\_register} anzulegen. Dieser Eintrag wurde von SDP selbstständig verwaltet und entfernt wenn die Sitzung zu SDP beendet wurde. Auf Grund des Wechsels zu DBus kann diese SDP-Schnittstelle nicht mehr genutzt werden, da sich keine Sitzungen zum SDP daemon aufbauen lassen, da dieser nicht mehr existiert.
        \begin{lstlisting}[frame=bt, label={lst:bluetooth:dbus}, language=C, caption=DBus Nutzung von SDP (Servercode in C)]
static DBusHandlerResult wrapper_messages(DBusConnection* connection, DBusMessage* message, void* user_data) {
    const char* interface_name = dbus_message_get_interface(message);
    const char* member_name = dbus_message_get_member(message);
    if (0==strcmp("org.bluez.Profile1", interface_name)) {
        if(0==strcmp("Release", member_name)) {
            profileRelease();
            return DBUS_HANDLER_RESULT_HANDLED;
        } else if(0==strcmp("NewConnection", member_name)) {
            profileNewConnection(message);
            return DBUS_HANDLER_RESULT_HANDLED;
        } else if(0==strcmp("RequestDisconnection", member_name)) {
            profileRequestDisconnection();
            return DBUS_HANDLER_RESULT_HANDLED;
        }
    }
    return DBUS_HANDLER_RESULT_NOT_YET_HANDLED;
}

DBusObjectPathVTable vtable;
vtable.message_function = wrapper_messages;
vtable.unregister_function = NULL;              

DBusConnection* conn = dbus_bus_get(DBUS_BUS_SYSTEM, &err);
dbus_connection_try_register_object_path(conn, PROFILE_PATH, &vtable, NULL, &err);

DBusMessage* msg = dbus_message_new_method_call("org.bluez", "/org/bluez", "org.bluez.ProfileManager1", "RegisterProfile");
...
dbus_connection_send_with_reply_and_block(conn, msg, -1, &err);
        \end{lstlisting}\footnote{TODO: Code Beispiel ist zu lang.}
        Um die SDP Funktionalität dennoch nutzen zu können, wird im \reflst{lst:bluetooth:dbus} gezeigt, wie ein Callback-Objekt registriert wird. Dieses Objekt hat die Aufgabe, ankommende Verbindungen zu akzeptieren und zu nutzen. Es ersetzt somit den akzeptierenden Serversocket. Weiterhin muss das Objekt in der Lage sein, bestehende Verbindungen schließen zu können. Dies hat zur Folge, dass die Komplexität im Vergleich zu einer Lösung ohne SDP stark erhöht wird, da nicht nur die DBus Nachrichten auf einem separaten Thread gehandhabt werden müssen, sondern auch eine Auflösung zwischen offenen Dateideskriptoren und den Client Kennungen stattfinden muss.
        Über diese gegebenen Dateideskriptoren konnte jedoch keine erfolgreiche Datenübertragung erzielt werden. Eine Verbindung wurde immer erfolgreich aufgebaut, jedoch schienen keine Daten tatsächlich übertragen zu werden, wodurch die Verbindung nach einem Timeout wieder geschlossen wurde.
        
        Ebenso war es serverseitig nicht möglich, die nötigen Sockets in pharo zu verwalten. Dem liegt zu Grunde, dass Sockets in der pharo VM über das SocketsPlugin\footnote{Quelle: https://github.com/pharo-project/pharo-vm/tree/master/mc/VMMaker.oscog.package/SocketPlugin.class} gekapselt verwaltet werden und dort fest als IP-Sockets erstellt werden.
        \begin{lstlisting}
newSocket = socket(AF_BLUETOOTH, SOCK_STREAM, BTPROTO_RFCOMM);

        \end{lstlisting}
    \subsection{NFC}
		
    \subsection{USB}

    \section{Ergebnisse der Evaluation}
		Durch die Betrachtung der Umsetzungsmöglichkeiten der einzelnen p2p Technologien kann nun eine Evaluation auf Grundlage der definierten Ziele vorgenommen werden. Zum Vergleich wird Wi-Fi Direct als bereits implementierte Technologie genutzt, um einen Bezug zur bestehenden Lösung aufbauen zu können.
		\subsection{Zuverlässigkeit}
		Die Robustheit der einzelnen Technologien bei Fehlern und die Reproduzierbarkeit von Verbindungen bilden zusammen das Kriterium der Zuverlässigkeit. Wie bei Wi-Fi Direct bereits deutlich wurde, sind diese beiden Punkte nicht zwingend für jede Technologie ausreichend gegeben. 
		\begin{itemize}
		\item {\it Wi-Fi Direct:} Eine hohe Fehleranfälligkeit in Kombination mit einer schlechten Fehlererholung durch Abstürze des laufenden Wi-Fi Daemon führen zu einer unzureichenden Zuverlässigkeit von Wi-Fi Direct. Ebenso konnte beobachtet werden, dass die internen Zustände der genutzten Bibliothek weder dokumentiert sind, noch waren diese Zustände sauber voneinander gekapselt, wodurch der Erfolg eines Verbindungsaufbaus von vorhergehenden Verbindungen abhängt. Die internen Zustandswechsel konnten zwar über Events der Bibliothek beobachtet werden, jedoch wurden daraus die Menge der möglichen Zustände und deren Übergänge nicht deutlich, da nicht zwingend jedes Event einem neuen Zustand entspricht. Die Dokumentation des {\it wpa\_supplicant} versäumt es ebenfalls zu beschreiben, in welcher Reihenfolge ein erfolgreicher Verbindungsaufbau vollzogen wird, wodurch nicht sichergestellt werden kann, dass die eigene Anbindung der erwarteten Nutzung entspricht.
		Zwar steht der Sourcecode der {\it wpa\_supplicant} Bibliothek und damit auch die Schnittstelle als Sourcecode zur Verfügung, da jedoch die Schnittstelle über Nachrichten statt über Methoden genutzt wird, ist es schwierig zu erkennen, welche Teile des Sourcecodes relevant sind. Es kommt hierbei erschwerend hinzu, dass nicht jede Bibliothek mit den gleichen Parametern gebaut wird, wodurch nicht zwingend der volle Funktionsumfang zur Verfügung steht.
		
		\item {\it Bluetooth:} Im Gegensatz dazu konnte mit Bluetooth und der genutzten Bibliothek ein Verbindungsaufbau mit reproduzierbaren Ergebnissen erreicht werden. Durch das integrierte Hilfsmittel {\it bluetoothctl} der BlueZ Bibliothek kann die Menge der bestehenden Zustände gut beobachtet werden. Die Benennung der einzelnen Zustände orientiert sich hierbei an der theoretischen Spezifikation von Bluetooth, wodurch auch die Zustandsübergänge bereits klar werden. Eine feinere Betrachtung der Zustände ist nicht nötig, da die Bluetooth Bibliothek jeglichen Zustand der Verbindung selbstständig verwaltet und dies der Anwendung als Serversocket zur Verfügung stellt. Fehler werden somit als Schließen des Sockets oder Fehler des Verbindungsaufbaus bekannt gemacht. Daraus lässt sich nicht der exakte Grund Ableiten lässt, jedoch besitzt Bluetooth aufgrund des saubere definierten Zustandsmodells und durch Socketverbindungen auch die Anbindung von Bluetooth eine solche Fehlerstabilität, dass die Verbindung lediglich neu aufgebaut werden muss \cite{bluetoothSpec}.
		
     \item {\it NFC:} Wenn ein bereits bestehender Wrapper genutzt wird, besteht das Problem, dass nicht ersichtlich ist, wann Fehler auftreten können. Die Kapselung auf Socketverbindungen für NFC ermöglicht es jedoch auch, eine Behandlung von Fehlern durch das Schließen bestehender Verbindungen durchzuführen und eine Verbindung erneut aufzubauen. Die internen Zustände eines NFC Verbindungsaufbaus lassen sich nicht direkt über die genutzte Bibliothek {\it libnfc} beobachten, jedoch kann über das Hilfsmittel {\it nfc-poll} ein NFC Gerät und die ausgetauschten Befehle ausgelesen werden. Eine Reproduzierbarkeit von Verbindungen ist durch das von NFC genutzte Trägersignal zwar gegeben, jedoch wird in der genutzten Implementierung ein hochfrequentes {\it Keep-Alive} Signal genutzt, damit die Verbindung aufrecht erhalten wird.
     
     \item {\it USB:} Ähnlich zu NFC schränkt die bestehende Implementierung von USB das Fehlerverhalten der zugrundeliegenden {\it libusb} Bibliothek ein und bündelt dieses ebenfalls über eine Socketverbindung. Da der Verbindungsaufbau von USB Geräten bereits vom Kernel übernommen wird, können Zustandsübergänge an neuen Geräten über {\it udev} beobachtet werden. Eine Fehleranfälligkeit der USB Verbindung ergibt sich aus den genutzten Treibern. Da jedoch das USB-Gerät ohne Treiber mit {\it libusb} angesprochen wird, besteht eine sehr geringe Fehleranfälligkeit einer USB Verbindung, da diese kabelgebunden ist und sich erst eine Fehleranfälligkeit aus den genutzten Treibern oder einer Abweichung von der USB Spezifikation ergibt \cite{usbSpec}. Die Spezifikation sieht eine Fehlererholung so vor, dass der Host der Verbindung laufende Datentransfers steuern kann. Diese Fehler werden erst propagiert, wenn sie nicht erfolgreich abgefangen werden konnten. Diese Fehler tauchen dann lediglich im Log des Kernels auf und resultieren in einem zurücksetzen der USB Verbindung über das USB Hotplugging.
		
		\end{itemize}
		
				
		\subsection{Wartbarkeit}
		Fehler in der Nutzung einer Technologie lassen sich nur dann leicht beheben, wenn die Dokumentation so vollständig ist, dass sie den betroffenen Nutzungsfall abdeckt und somit eine Abweichung der Anbindung von der Dokumentation aufgedeckt werden kann. Sollte die Dokumentation jedoch nicht ausreichend auf diesen Fall eingehen, muss der Quellcode der angebotenen Schnittstelle zur Verfügung stehen und leicht zu verstehen sein, sodass die Abweichung leicht gefunden werden kann. Sollten Tests für die genutzten Bibliotheken im Sourcecode vorhanden sein, verbessert dies die externe Wartbarkeit ebenfalls, da so zum Einen bereits eine Abstraktion der Hardware besteht und diese für eigene Tests genutzt werden kann und zum Anderen die Nutzung der Bibliothek anhand der getesteten Nutzungsfälle exemplarisch gezeigt wird.
		\begin{itemize}
		\item {\it Wi-Fi Direct:} Die Dokumentation von Wi-Fi Direct ist bei der Wi-Fi Association nicht öffentlich zugänglich, jedoch enthält die Beschreibung der {\it wpa\_supplicant} Implementierung eine Aufstellung einiger Ereignisse, die bei der Nutzung auftreten können. Fehler beim Verbindungsaufbau oder während der Verbindungsnutzung werden zwar als solche Ereignisse der Anbindung mitgeteilt, jedoch ließen sich auch eine mangelnde Fehlerrobustheit durch Abstürze bei einigen dieser Ereignisse feststellen. Tests sind im {\it wpa\_supplicant} für die einzelnen Schritte eines p2p Verbindungsaufbaus als Python-Skripte vorhanden und basieren auf einer simulierten Hardware. Diese simulierte Hardware lässt sich somit auch für eigene Tests nutzen, um die eigene Implementierung mit möglichst geringen Abänderungen testen zu können.
		
		\item {\it Bluetooth:} Eine Anbindung an Bluetooth geschieht über Sockets, welche bei der Erstellung durch das Austauschen einiger Parameter bereits eine testbare Umgebung bereitstellen können. Weiterhin wird die Spezifikation von Bluetooth durch die Bluetooth SIG öffentlich einsehbar bereitgestellt und beschreibt ausführlich interne Zustände einer Bluetooth Implementierung sowie die Testbarkeit eines Bluetoothgerätes\cite{bluetoothSpec}. Die BlueZ Bibliothek steht ebenfalls als Referenz zur Verfügung und enthält in den Tests ebenfalls eine Beispielanbindung für die Nutzungsfälle der Bibliothek jedoch keine Virtualisierung der Hardware. Dies bedeutet, dass für einen vollständigen Integrationstest mindestens zwei Bluetoothschnittstellen am testenden Gerät vorhanden sein müssen, da ein Loopback ebenfalls nicht möglich ist.
		
		\item {\it NFC:}
     \item {\it USB:}
		\end{itemize}
		\subsection{Funktionalität}

      \begin{itemize}
		\item {\it Wi-Fi Direct:}
		\item {\it Bluetooth:}
		\item {\it NFC:}
     \item {\it USB:}
		\end{itemize}
		
		\subsection{Effizienz}

      \begin{itemize}
		\item {\it Wi-Fi Direct:}
		\item {\it Bluetooth:}
		\item {\it NFC:}
     \item {\it USB:}
		\end{itemize}
		
		\subsection{Bedienbarkeit}

      \begin{itemize}
		\item {\it Wi-Fi Direct:}
		\item {\it Bluetooth:}
		\item {\it NFC:}
     \item {\it USB:}
		\end{itemize}
		
		\subsection{Zusammenfassung}
%    \begin{figure}[ht]
%    \centering
%    \begin{tikzpicture}
%        \tkzKiviatDiagram[scale=1.0,label distance=.25cm, radial = 5, gap = 1, lattice = 5]{Zuverlässigkeit,Wartbarkeit,Funktionalität,Effizienz,Bedienbarkeit}
%        %Wi-Fi Direct:
%        \tkzKiviatLine[thick,color=red,mark=none, fill=red!20,opacity=.5](2,2,4,4,3)
%        %Bluetooth:
%        \tkzKiviatLine[thick,color=blue, fill=blue!20,opacity=.5](4,4,4,2,4)
%        %NFC:
%        \tkzKiviatLine[thick,color=yellow, fill=red!20,opacity=.5](4,3,3,1,2)
%        %USB:
%        \tkzKiviatLine[thick,color=green, fill=green!20,opacity=.5](1,3,3,3,1)
%    \end{tikzpicture}
%    \caption{M1} \label{fig:rating}
%    \end{figure}
    \section{Fazit}

Die Überprüfung von Softwarequalität auf einer Kombination aus Technologie und Software stellt die Herausforderung, bewertbare Ergebnisse zu erhalten. Da die Qualitätsmerkmale der Zuverlässigkeit, Wartbarkeit, Funktionalität und Bedienbarkeit lediglich auf Grund von Erfahrungswerten bewertet wurden und ihre Spezifikation wesentlich zu komplex sind, um sie im Rahmen dieser Arbeit vollständig auszuwerten, kann nur eine informale Aussage über die eingesetzten Technologien getroffen werden. Jede der Technologien setzt unterschiedliche Schwerpunkte bei den überprüften Qualitätskriterien, wodurch jede der Technologien für manche Anwendungsfälle in Frage kommt. Der Anwendungsfall einer Verbindungskonfiguration als Client-/Server-Lösung zeichnet sich durch ein geringes Datenvolumen in Verbindung mit teilweise hohen Antwortzeiten auf Anwendungsebene aus. Auf Grund dieser Gegebenheiten wird der Evaluationspunkt der Effizienz übergangen und lediglich anhand der verbleibenden Evaluationspunkte ein Ergebnis der Evaluation gezogen.

Für jede der genutzten Technologien kommt ein HTTP Wrapper zum Einsatz, welcher es ermöglicht, die HTTP Befehle über eine andere Technologie als Wi-Fi zum Server zu übertragen. Dieser Wrapper ist eine simple Lösung, die jegliche Technologie als Sockets abbildet, um eine simple Auswechselbarkeit der Technologien zu ermöglichen. Dies hilft ebenfalls das gesetzte Ziel einer Alternative zu Wi-Fi Direct zu erfüllen, da sich die unterschiedlichen Technologien mit geringem Aufwand parallel in der Anwendung nutzen lassen und so dem Nutzer auch eine Wahl zwischen mehreren Verbindungsarten gelassen werden kann. Ebenso müssen keine Änderungen an den Anbindungen der Technologien vorgenommen werden, wenn die Schnittstelle zur Verbindungskonfiguration erweitert werden soll. Da alle Anbindungen von Technologien die selbe Grundlage nutzen, um sie im Wrapper nutzen zu können, entsteht kein zusätzlicher Wartungsaufwand der existierenden Anbindungen, wenn eine neue Technologie ebenfalls genutzt werden soll.

Während Umsetzung wurde deutlich, dass nicht nur Wi-Fi Direct \linebreak Unzulänglichkeiten in seiner Implementierung besitzt. Alle getesteten Technologien weisen in unterschiedlichen Punkten Probleme auf, die jedoch nicht zwingend die Nutzbarkeit für eine Verbindungskonfiguration beeinträchtigen. Die aufgetretenen Probleme verdeutlichen jedoch noch einmal die unterschiedlichen Ansätze der Technologien.

Bluetooth mangelt es auf Grund des Wechsels der Art von Bibliotheks-API an Dokumentation und allgemeinen Beispielen über die Nutzung der Bibliothek. Gleichzeitig traten Probleme bei der Nutzung von {\it SDP} auf, wodurch eine vereinfachte Lösung über einen expliziten RFCOMM Port genutzt werden muss. Dies hat jedoch den Vorteil, dass die Anbindung von Bluetooth wesentlich simpler ist, da keine DBus-API angesprochen werden muss.

Für eine Nutzung von NFC zeigt sich eine Unzulänglichkeit im Bereich der Benutzbarkeit, da der Nutzer gezwungen ist, das Gerät mehrmals in die direkte Nähe des zu konfigurierenden Gerätes bringen, was je nach Einsatzort keine akzeptable Lösung darstellt. Ebenso basiert NFC auf dem Austausch von kurzen Nachrichten über ein Trägersignal und unterstützt so lediglich eine indirekte Kommunikation  beider Teilnehmer durch das Reagieren auf Befehle, was die Funktionalität der Lösung ein wenig dämpft, da so eine bidirektionale Kommunikation zusätzliche Komplexität birgt.

Eine Verwendung von USB scheitert zur Zeit daran, dass eine AOA Verbindung nicht korrekt aufgebaut werden kann dem kann zu Grunde liegen, dass AOA seit dem Jahr 2012 von Google nicht mehr aktiv weiter entwickelt wurde und inzwischen Fehler in der Anbindung dieses Protokolls existieren. Andererseits ist es auch möglich, dass zusätzlicher undokumentierter Konfigurationsaufwand unter Android vorgenommen werden muss. Da keine erfolgreiche Datenübertragung über USB vollzogen werden konnte, kann diese Technologie auch nicht im Rahmen einer Verbindungskonfiguration genutzt werden.

NFC ist gegenüber USB und Wi-Fi Direct ebenfalls eine bessere Alternative, um p2p Verbindungen zu ermöglichen. Zwar liegen die Bedienbarkeit und Funktionalität nicht auf dem Niveau von Bluetooth, da NFC eine für den Nutzer und die Implementierung umständlichere Technologie nutzt, jedoch werden diese wieder durch die Wartbarkeit und Zuverlässigkeit des Mediums aufgewogen.

In der Evaluation hat sich so Bluetooth als beste Technologie für eine p2p Verbindung durchgesetzt, weil sie die für eine Verbindungskonfiguration relevanten Kriterien den anderen Technologien gegenüber besser erfüllt. Besonders die Bedienbarkeit durch den Nutzer zusammen mit einer sehr guten Wartbarkeit geben Bluetooth einen großen Vorteil gegenüber den verbleibenden Übertragungsmedien. Die Bedienbarkeit stellt das wichtigste Kriterium  der geprüften Softwarequalität dar, da die anderen Merkmal nicht zwingend vom Nutzer wahrgenommen werden und eine Software ohne Nutzer auch keinen Nutzen besitzt.

\subsection{Ausblick?}


    \addcontentsline{toc}{section}{\lstlistlistingname}
    \lstlistoflistings
    \addcontentsline{toc}{section}{\listfigurename}
    \listoffigures
    \pagebreak
    
    \addcontentsline{toc}{section}{Literaturverzeichnis}
    \renewcommand\refname{Literaturverzeichnis}
    \begin{thebibliography}{10}
        \bibitem[Aneja et al.]{Aneja}Nagender Aneja und Sapna Gambhir: "Profile-Based Ad Hoc Social Networking Using Wi-Fi Direct on the Top of Android" {\it Mobile Information Systems, Volume 2018, Article ID 9469536, 7 pages} (2018).
        \bibitem[Eberhardt et al.]{Kelm} Udo Eberhardt und Hans Joachim Kelm [Hrsg.]: {\it USB - Universal Serial Bus} Franzis, Poing (1999).
        \bibitem[Esnaashari et al.]{Esnaashari} Shadi Esnaashari, Ian Welch und Peter Komisarczuk: {\it Determining home users' vulnerability to Universal Plug and Play (UPnP) attacks} Erschienen in: Proceedings of the 2013 27th International Conference on Advanced Information Networking and Applications Workshops (WAINA). IEEE, 2013. - S. 725-729. - ISBN 9781467362399 .
        \bibitem[Kaiser et al.]{Kaiser} Daniel Kaiser, Marcel Waldvogel, Holger Strittmatter und Oliver Haese: {\it User-Friendly, Versatile, and Efficient Multi-Link DNS Service Discovery} Erschienen in: Proceedings 2016 IEEE 36th International Conference on Distributed Computing Systems Workshops : ICDCSW 2016. - Piscataway, NJ : IEEE, 2016. - S. 146-155. - ISBN 978-1-5090-3686-8.
        \bibitem[Kaiser, Waldvogel]{Kaiser2} Daniel Kaiser und Marcel Waldvogel: {\it Adding Orivacy to Multicast DNS Service Discovery} Erschienen in: Proceedings - 2014 IEEE 13th International Conference on Trust, Security and Privacy in Computing and Communications (TrustCom) ; Beijing, China, 24 Sep - 26 Sep 2014. - Piscataway : IEEE, 2014. - S. 809-816. - ISBN 978-1-4799-6513-7.
        \bibitem[Langer et al.]{Langer}Josef Langer und Michael Roland: {\it Anwendungen und Technik von Near Field Communication (NFC)} Springer, Heidelberg (2010).
        \bibitem[Lüders]{Lueders}Christian Lüders: {\it Lokale Funknetze: Wireless LANs (IEEE 802.11), Bluetooth, DECT} Vogel, Würzburg (2007).
        \bibitem[Morrow]{Morrow}Robert Morrow: {\it Bluetooth Operation and Use} McGraw-Hill (2002).
        \bibitem[Sauter]{Sauter}Martin Sauter: {\it Grundkurs Mobile Kommunikationssysteme: LTE-Advanced, UMTS, HSPA, GSM, GPRS, Wireless LAN und Bluetooth} 6.Auflage Springer Vieweg, Wiesbaden (2015).
        \bibitem[Sikora]{Sikora}Axel Sikora: {\it Wireless LAN: Protokolle und Anwendungen} Addison-Wesley, München u.A. (2001).
		\bibitem[Kretzschmar]{aiProject} Jan Phillip Kretzschmar: {\it Verbindungskonfiguration von PharoThings auf Raspberry Pi durch Android App} TH Köln, Anhang (2019).
    \end{thebibliography}
    \pagebreak
    
    \addcontentsline{toc}{section}{Internetquellen}
    \renewcommand\refname{Internetquellen}
    \begin{thebibliography}{9}
        \bibitem[Android Bluetooth]{androidBluetooth}\texttt{https://developer.android.com/guide/topics/\linebreak connectivity/bluetooth\#ConnectAsAServer}
        \bibitem[Android Open Accessory]{aoa}\texttt{https://source.android.com/devices/\linebreak accessories/custom}
        \bibitem[AOA Proxy Server]{aoaProxyAccessory}\texttt{https://github.com/timotto/AOA-Proxy/blob/\linebreak master/src/accessory.c} (Zeilen ?)
        \bibitem[Bluetooth Spezifikation]{bluetoothSpec}{\texttt https://www.bluetooth.com/specifications/bluetooth-core-specification/}
        \bibitem[BlueZ 5 Porting]{bluezMigration}\texttt{http://www.bluez.org/bluez-5-api-introduction-and-\linebreak porting-guide/}
        \bibitem[BlueZ SDP]{bluezRfcomm}\texttt{https://people.csail.mit.edu/albert/bluez-intro/x148.html}
        \bibitem[okhttp RealCall]{okhttpRealCall}\texttt{https://github.com/square/okhttp/blob/okhttp\_3.11.x/\linebreak okhttp/src/main/java/okhttp3/RealCall.java} (Zeilen 185-194)
        \bibitem[ITU Study Group]{ituGroup}\texttt{https://www.itu.int/en/ITU-T/about/groups/Pages/sg20.aspx}
        \bibitem[libusb]{libusb}\texttt{http://libusb.sourceforge.net/api-1.0/index.html}
        \bibitem[Pharo Socket Plugin]{pharoSocket}\texttt{https://github.com/pharo-project/pharo-vm/tree/\linebreak master/mc/VMMaker.oscog.package/SocketPlugin.class}
        \bibitem[Microservices]{microservices}\texttt{https://microservices.io/}
        \bibitem[MIT ADK]{mitADK}\texttt{https://stuff.mit.edu/afs/sipb/project/android/docs/tools/\linebreak adk/adk2.html}
        \bibitem[NFC Google Bug]{nfcBug}\texttt{https://issuetracker.google.com/issues/37005118}
        \bibitem[NFC Rates]{nfcRates}\texttt{https://nfc-forum.org/resources/what-are-the-data-\linebreak transmission-rates/}
        \bibitem[NFC Sockets Client]{nfcSocketsClient}\texttt{https://github.com/classycodeoss/nfc-sockets/blob/\linebreak master/android/NFCSockets/app/src/main/java/com/classycode/\linebreak nfcsockets/sockets/NFCSocket.java} (Zeilen ?)
        \bibitem[NFC Sockets Server]{nfcSockets}\texttt{https://github.com/classycodeoss/nfc-sockets/blob/\linebreak master/pn532\_socket\_tunnel/pn532\_socket\_tunnel.c} (Zeilen ?)
        \bibitem[NFC Sockets Blog]{nfcSocketsBlog}\texttt{blog.classycode.com/sockets-over-nfc-on-\linebreak android-c294b6c58bbf}
        \bibitem[NFC-Tools libnfc]{libnfc}\texttt{https://github.com/nfc-tools/libnfc}
        \bibitem[NXP PN532 Datenblatt]{nxpChip}\texttt{https://www.nxp.com/docs/en/nxp/data-sheets/\linebreak PN532\_C1.pdf}
        \bibitem[USB Spezifikation]{usbSpec}\texttt{https://usb.org/documents}
        \bibitem[Wi-Fi Direct Rates]{wifiRate}\texttt{https://www.wi-fi.org/knowledge-center/faq/\linebreak how-fast-is-wi-fi-direct}
    \end{thebibliography}
    \pagebreak
    \addcontentsline{toc}{section}{Anhänge}
    \appendix
    \stopcontents

    \startcontents[sections]
    %\begin{titlepage}
    \section{Anhang: Verbindungskonfiguration von PharoThings auf Raspberry Pi durch Android App}
    \includegraphics[width=0.4\textwidth]{../latex-ai-project/th_logo.png}
    ~\\[2.5cm]
    \begin{center}
    \textbf{\huge Verbindungskonfiguration von PharoThings auf Raspberry Pi durch Android App}\\[0.5cm]
    {\Large Praxisprojekt Sommersemester 2019}
    \vfill
    \end{center}
    ~\\[2.0cm]
    \begin{flushright}
    {\large Jan Phillip Kretzschmar \it{(jan@2denker.de)}}\\[0.1cm]
    ~\\[1.0cm]
    {\large Betreuer (Zweidenker GmbH):}\\[0.1cm]
    {\large Christian Denker \it{(christian@2denker.de)}}
    ~\\[0.5cm]
    {\large Betreuer (TH Köln):}\\[0.1cm]
    {\large Prof. Christian Kohls}\\[0.1cm]

	~\\[1.0cm]
    {\large 30. Juni 2019}
	\end{flushright}
    %\end{titlepage}

    \pagebreak
    \section*{\contentsname}
    \printcontents[sections]{ }{2}{}
    \pagebreak

    \begin{leveldown}
    \section{Expose}
    Pharo ist eine auf Smalltalk basierende objektorientierte und dynamisch getypte Programmiersprache,
    welche gleichzeitig ihre eigene live Entwicklungsumgebung mit mächtigen Debugging-Tools ist.\footnote{http://pharo.org/}
    PharoThings bietet eine reduzierte Platform für  Internet of Things(IoT), sodass eine Ausführung von Programmen auf Kleinstcomputern möglich ist.
    Um keine Kompromisse im Bereich der Entwicklungsumgebung machen zu müssen, kann über Remote Debugger auf live Programme zugegriffen werden.
    Dadurch ist lediglich die Ausführung auf das IoT Gerät ausgelagert. PharoThings bietet in Verbindung mit WiringPi eine Platform für Raspberry Pi,
    auf der Board Modeling simpel möglich ist.\footnote{https://github.com/pharo-iot/PharoThings}
    Für die Erstkonfiguration und Verbindungskonfigurationvon PharoThings auf Raspberry Pi ist es aktuell nötig,
    diese Konfiguration z.B. der WLAN-Verbindung durch einen Computer vorzunehmen.
    Um die Pharo Things Laufzeitumgebungen als IoT-Geräte simpel nutzen zu können, ist es nötig,
    die Konfiguration der Installationen und Geräte zu vereinfachen. Dabei soll eine Android App diesen Vorgang übernehmen:
    \begin{enumerate}
        \item Erkennen und Auflisten von IoT-Geräten in der Nähe. Die zu verwendende Kommunikationstechnologie ist dabei zu evaluieren.
        \item Verbindungsaufbau zu ausgewähltem IoT-Gerät
        \item IoT-Gerät erhält Hostnamen, WLAN-Konfiguration, Beacon Intervall, etc.
        \item Eventuelle Verbindungsprobleme der erst WLAN-Verbindung werden über die bestehende Verbindung zurückgemeldet
        \item Pharo Things-Installationen im aktuellen WLAN werden aufgelistet.
    \end{enumerate}
    Um diesen Vorgang umsetzen zu können, werden drei Komponenten implementiert:
    \begin{enumerate}
        \item Ein Protokoll muss definiert werden, welches die Kommunikation zu PharoThings Instanzen zur Konfiguration festlegt.
        Weiterhin muss festgelegt werden, in welcher Art und Weise ein Beacon-Signal im WLAN von den Installationen gesendet wird.
        Um das WLAN nicht zu überlasten, empfiehlt es sich diese Nachrichten kurz zu halten. Es ist zu evaluieren,
        ob sich Installationen auch gegenseitig erkennen können, sodass ein gebündelter Beacon gesendet werden kann.
        \item Eine Android App, welche den Nutzer durch den beschriebenen Vorgang leitet, muss implementiert werden.
        Der Fokus hierbei liegt darin, diesen Vorgang mit möglichst wenig Nutzerinteraktion durchzuführen.
        \item Eine Anwendung in PharoThings muss erstellt werden, welche das Protokoll implementiert
        und basierend darauf sich in einem WLAN einwählen kann und ein Beacon-Signal in diesem WLAN sendet.
    \end{enumerate}
    Das Projekt wird mit Unterstützung der Zweidenker GmbH durchgeführt.
    \section{Mögliche Kommunikationstechnologien}
    Ein Ad Hoc Netzwerk bietet im Allgemeinen die Möglichkeit {\it peer to peer} (p2p) Verbindungen zwischen Geräten dezentralisiert aufzubauen.
    Geräte können hierbei selbstständig eine Netzwerkverbindung untereinander aushandeln. Da solche Verbindungen nur dann sinnvoll sind,
    wenn es Daten gibt, die nur zwischen den beiden verbundenen Geräten ausgetauscht werden müssen, ergibt ein solches Netzwerk meist nur
    im Bezug auf eine tatsächlliche Anwendung Sinn. Die erweiterte Definition des Ad Hoc Netzwerks
    bezieht somit alle Netzwerkschichten des OSI-Modells mit ein.\cite[S.23]{Sikora}
    Obwohl das OSI-Modell vor Allem auf Ethernet und WLAN ausgelegt ist, lässt sich die Definition des Ad Hoc Netzwerks
    dennoch für weitere Kommunikationstechnologie übernehmen, da diese ebenfalls p2p Verbindungen aufbauen können.
    Optimalerweise sollte es möglich sein die bestehenden WLAN-Verbindungen beider Geräte während einer p2p Verbindung beibehalten zu können.
    Für Kommunikationsmedien in diesem Projekt fallen einige Beschränkungen an: 
    \begin {enumerate}
    \item {\it Reichweite:}
    Netzwerke werden oft nach ihrer Reichweite klassifiziert. Dabei gibt es die geläufigen Bezeichnungen Local Area Network (LAN),
    Metropolitan Area Network, Wide Area Network und Global Area Network, die in ihrer Klassifizierung von Gebäuden zu einer Globalen Reichweite übergehen.
    Üblicherweise verbinden höhere klassifizierte Netzwerke niedriger klassifizierte Netzwerke miteinander. Im Bereich der Drahtlosnetzwerke gibt es jede
    dieser Klassen ebenfalls als Drahtlos-Variante: WLAN, WMAN, WWAN und WGAN, es kommen jedoch noch zwei in ihrer Reichweite kleinere Netzwerke hinzu,
    das Wireless Body Area Network und das Wireless Personal Area Network. Übliche Einsatzgebiet des WBAN sind im medizinischen Bereich zu finden,
    aber auch Near Feald Communication fällt in diese Kategorie. Unter die Klassifizierung des WPAN fällt unter anderem Bluetooth,
    für dieses Projekt sind somit Funknetzwerke der untersten drei Kategorien WBAN, WPAN und WLAN oder deren kabelgebundenen Derivate interessant.\cite[S.17]{Lueders}
    \item {\it Unterstützung in Android Smartphones:}
    Damit eine große Anzahl an potentiellen Nutzern angesprochen werden kann, muss die Verbindungsschnittstelle von Smartphones unterstützt werden.
    Für dieses Projekt wird dabei nur Android betrachtet.
    Aktuelle Smartphones bieten im Allgemeinen zur Zeit die vier Schnittstellen {\bf USB, NFC, Bluetooth und Wi-Fi},
    über die sich Verbindungen zu Geräten in der näheren Umgebung aufbauen lassen.
    \item {\it Hardware an IoT Geräten:}
    Als IoT Gerät dient in diesem Projekt ein Raspberry Pi.
    In den Varianten {\it Model 3 B, Model 3 B+ and Model Zero W} bietet Dieser USB, Bluetooth und Wi-Fi als mögliche Schnittstellen.
    Durch die Verwendung der GPIO-Pins ist es außerdem möglich, ein NFC-Modul anzubinden,
    jedoch würde Dies die später nutzbaren Pins unerwünscht einschränken.
    Weiterhin bietet der Raspberry Pi ein vollständiges Betriebssystem mit Benutzeroberfläche, jedoch soll eine Internetverbindung
    ohne Peripherie am Raspberry Pi konfiguriert werden können. 
    \end {enumerate}

    \subsection{Kommunikation über WLAN}
        Der IEEE802.11 Standard siedelt sich im OSI-Modell lediglich in der Physical Layer und Data Link Layer an. Ihr eigentlicher Sinn ist es,
        IP-Pakete der Network Layer im gleichen Maße wie ein LAN übertragen zu können.
        Die Definition des Wireless LAN unterscheidet sich jedoch vom LAN Standard dahingehend, dass eine vollständig eigene Physical Layer geschaffen wurde,
        da das Übertragungsmedium andere Restrikitionen besitzt. Die Data Link Layer setzt sich für WLAN größtenteils aus drei Teilen zusammen.
        Die Logic Link Control nach 802.2 und das Bridging nach 802.1 sind mit LAN identisch, um der Network Layer eine einheitliche Schnittstelle unabhängig des Übertragungsmediums zu bieten.
        In der Data Link Layer unterscheidet sich lediglich der Media Access Control (MAC).\cite[S.311]{Sauter}
        Dieser regelt im Fall von WLAN den Zugriff auf das Übertragungsmedium durch unterschiedliche Wartezeiten zwischen Frames und die Reservierung des Mediums zum Senden von Frames.
        Da das MAC-Protokoll zudem die Addressierung von Geräten ermöglicht, bietet es ebenfalls bereits die Möglichkeit, Broadcasts zu senden.
        Um die hohe Fehleranfälligkeit eines Drahtlosnetzwerkes für höhere Schichten zu reduzieren, wird jedes Frame vom Empfänger bestätigt.\cite[S.325-327]{Sauter}
        
        Ein Netzwerk nach 802.11 kann hierbei entweder im Infrastruktur Modus, in dem alle Geräte ausschließlich mit einem Access Point kommunizieren,
        oder im Ad Hoc Modus, welcher die direkte Kommunikation zwischen Geräten erlaubt, betrieben werden.\cite[S.82]{Sikora}
        Ein Verbindungsaufbau ist für eine p2p Verbinung entweder durch eine konkrete Implementierung des Ad Hoc Modus oder
        im MAC-Protokoll zu ersuchen.

        Unter dem Markennamen Wi-Fi\textsuperscript{TM} werden 802.11-kompatible Geräte zertifiziert.\cite[S.80]{Sikora}
        Für den Ad-hoc Modus nach 802.11 wurde dabei Wi-Fi Peer-to-Peer (Wi-Fi Direct)\textregistered\cite{wifiDirect} als ein universeller Standard definiert.
        Durch Wi-Fi Direct ist es jedoch ebenfalls möglich, einen Verbindungsaufbau in der {\bf Application Layer} des OSI-Modells anzusiedeln.
        Dazu bietet diese Spezifikation neben dem normalen Peer-To-Peer Modus die Möglichkeit, Services anzubieten und zu finden, bevor eine Verbindung etabliert werden muss.
        Grundlage für diese Services bilden dabei DNS Service-Discovery (DNS SD) und Universal Plug and Play (UPnP).
        DNS SD ist die Weiterentwicklung von Apple Bonjour und wird größtenteils genutzt um eine Zero-Configuration Service Discovery von beispielsweise Netzwerkdruckern zu ermöglichen. Da UPnP im Gegensatz zu DNS SD auch die Kontrolle über die Services übernimmt, besitzt es einige Verwundbarkeiten. Man kann es jedoch ähnlich zu DNS SD lediglich dazu benutzen, Services zu ernennen.\cite{Esnaashari}
    
        \paragraph{Nutzung unter Android}
        Im Gegensatz zu Apple AirPlay kann für eine App nicht auf Gerätetreiber-Ebene entwickelt werden. Um MAC in Linux verwenden zu können, ist es nötig, Sockets mit dem Attribut {\bf SOCK\_RAW} zu öffnen,
        um eigene MAC-Pakete senden zu können. Solche Sockets können jedoch nur mit der Berechtigung {\bf CAP\_NET\_RAW} erstellt werden.\cite{linuxPacket}
        Unter Android fällt diese Berechtigung mangels Granularität root zu, wodurch diese Lösung unpraktikabel wird,
        wenn eine möglichst große Nutzergruppe angesprochen werden soll.\cite{androidRights}
        Android bietet jedoch ab API 14 die Möglichkeit, sich über WI-Fi Direct als möglicher peer anderen Geräten zu präsentieren und
        p2p Verbindungen aufzubauen, sowie Services bereitzustellen und zu erkennen.
        Android stellt diese p2p Funktionalität als {\it WifiP2PManager} bereit. Ein kurzer Test mit zwei Android Geräten hat dabei ergeben,
        dass dieser bestehende Wi-Fi-Verbindungen während der Service Discovery beibehält.\cite{test-repository}
        Ein ähnlicher Ansatz, in dem Wi-Fi Direct genutzt wird, um ein Ad Hoc Netzwerk aufzubauen findet sich in \cite{Aneja}.

    \subsection{Kommunikation über Bluetooth}
        Die grundlegenden Komponenten von Bluetooth sind im Standard IEEE802.15.1 als {\it Bluetooth Core} definiert. Anwendungen können über diesen Kern oder speziellere Protokolle Bluetooth
        Verbindungen zu anderen Geräten aufbauen.\cite[S.228]{Lueders}
        Um Daten über Bluetooth senden zu können, muss eine Verbindung zwischen den beiden Geräten aufgebaut werden. Für diese Verbindung ist es nötig,
        dass sich die entsprechenden Geräte zunächst koppeln. Zum Koppeln verfügbare Geräte werden hierbei durch eine sogenannte Inquiry, welche erreichbare Geräte auffordert,
        sich zu identifizieren, aufgelistet. Da nun die Adresse des zu koppelnden Gerätes bekannt ist, kann zu diesem eine Koppelung angefragt werden, was als Paging bezeichnet wird.
        Die Kopplung dient dabei dem Austausch der Frequency Hop Sequenzen, welche festlegt, wann auf welcher Frequenz Pakete gesendet werden, sowie dem Pairing von Geräten, welches sicherstellt,
        dass das richtige Gerät gekoppelt wird und Schlüssel zur Authentifizierung und Verschlüsselung überträgt.\cite[S.402f.]{Sauter}

        Sobald eine Verbindung zwischen den Geräten aufgebaut wurde, können Daten bidirectional übertragen werden, indem ein Bluetoothkanal, anders als WLAN, in Zeitslots unterteilt wird.
        Die alleinige Kontrolle, welches Gerät wann Daten senden darf, hat dabei das Mastergerät des Netzes. Das Gerät, welches den Verbindungsaufbau angefragt hat,
        agiert dabei als Master und bis zu sieben weitere Geräte können als Slave in einem Piconetz verbunden sein.\cite[S.379f.]{Sauter}

        Für bestimmte Anwendungsfälle gibt es spezielle Protokolle, welche generische Lösungen dieser Fälle bieten.
        So ermöglicht das Service Discovery Protocol (SDP) den Informationsaustausch über verfügbare Dienste der Kommunikationspartner,
        Radio Frequency Communications (RFCOMM) kann serielle Schnittstellen abbilden und das Object Exchange Protocol (OBEX) kann Datenobjekte über RFCOMM übertragen.\cite[S.229]{Lueders}
        SDP erlaubt nicht nur das Auflisten von verfügbaren Diensten des Verbindungspartners, sondern auch das Suchen von einem bestimmten Dienst auf erreichbaren Geräten,
        es kann dabei jedoch keinen Zugriff zu den Diensten bereitstellen oder die Verfügbarkeit der Dienste auf den erreichbaren Geräten angeben.\cite[S.395f]{Morrow}

        Bluetooth lehnt sich mit dem {\it Bluetooth Core} nur lose an das OSI-Referenzmodell an, da nicht jedes Element des Bluetoothstacks sich sauber einer OSI-Schicht zuordnen lässt.
        Die oben vorgestellten Protokolle wie SDP oder RFCOMM siedeln sich jedoch in der {\bf Presentation Layer} an, da diese lediglich dazu gedacht sind,
        Daten leichter der Anwendung zur Verfügung stellen zu können.\cite[S.382]{Sauter}

        \paragraph{Bluetooth Low Energy}
        Als Teil des Bluetooth Standards ist Bluetooth Low Energy (LE) darauf ausgelegt, Daten mit einem möglichst geringen Energieverbrauch über die Bluetooth-Hardware zu übertragen.
        Dies wird ermöglicht, indem verbindungslose Broadcast, die Nutzerdaten enthalten können definiert werden, sowie aktive Verbindungen zwischen zwei Geräten auf eine kurze Übertragung von 6 Packeten pro Verbindungsevent limitiert werden. Verbindungsevents sind dabei durch ein wählbares Intervall zwischen 7.5ms und 4s von einander getrennt. Weiterhin beschränkt sich Bluetooth LE im Hinblick auf die Reichweite darauf, möglichst wenig Energie zu verbrauchen, da eine hohe Sendeleistung in einem hohen Energieverbrauch resultiert. Da sich die Sendeleistung jedoch pro Gerät konfigurieren lässt, ist die theoretische Reichweite ähnlich zu Bluetooth gegeben.\cite[S.7f.]{Townsend}

        \paragraph{Nutzung unter Android}
        Android stellt sowohl Bluetooth als auch Bluetooth LE als {\it BluetoothAdapter} bereit. Ähnlich zu Wi-Fi Direct  Android bietet ab API 5 bzw. API 18 die Möglichkeit, über Bluetooth als möglicher peer anderen Geräten zu präsentieren und
        p2p Verbindungen aufzubauen. Über Bluetooth LE können hingegen Services als Broadcasts bereitgestellt werden. Bluetooth in Verbindung mit Bluetooth LE lässt sich somit im Rahmen dieser Arbeit ohne Probleme auf Android benutzen.

    \subsection{Kommunikation über NFC}
        Die Near Field Communication (NFC)-Technologie basiert auf RFID-Systemen und erlaubt die Übertragung von Daten auf eine Distanz bis zu 10 Zentimeter.
        Grundlegend ist NFC in den Standards NFCIP-1 (ECMA-340) und NFCIP-2 (ECMA-352) definiert. Wie bei klassischen RFID-Systemen ist ein magnetisches Feld
        als Trägersignal der Daten definiert. Dieses magnetische Feld dient dazu, passiven Komponenten, die keine eigene Stromversorgung besitzen, auf Anfragen
        der steuernden Komponente zu antworten. Anders als RFID hebt NFC jedoch die strikte Trennung zwischen steuernder (Initiator) und gesteuerter (Target) Komponente auf, sodass
        jedes teilnehmende NFC-Gerät zumindest theoretisch beide Rollen übernehmen kann.\cite[S.89]{Langer} Rückwärtskompatibilität erreicht NFC,
        indem sowohl ein Reader-Writer-Modus, der die Kommunikation mit passiven RFID-Transpondern ermöglicht,
        als auch ein Card-Emulation-Modus, der die Kommunikation mit RFID-Lesegeräten erlaubt, definiert ist.\cite[S.99f.]{Langer}
        Als letzter Modus ist der hier relevante Peer-to-Peer-Modus, dazu in der lage, Daten zwischen zwei NFC-Geräten zu übertragen.
        Ähnlich zu Bluetooth lehnt sich auch NFC nur lose an das OSI-Modell an. Es gibt für NFC drei aufeinander aufbauende Protokollschichten.
        Die Bitübertragung ist in NFCIP-1 definiert und teilt sich in einen aktiven und passiven Modus. Der aktive Modus hebt sich dabei vom passiven Modus,
        welcher in seiner Funktionsweise äquivalent zum RFID-System ist, dadurch ab, dass das Trägersignal von beiden Kommunikationspartnern abwechselnd generiert wird.
        Die darauffolgende Schicht ist ein MAC Protokoll, welches den Zugriff auf das Übertragungsmedium sichert, indem geprüft wird, ob bereits ein Trägersignal existiert,
        bevor ein Gerät in den Initiator-Modus wechselt, andernfalls verbleibt es im Target-Modus.
        Die letzte Schicht bildet das Logic Link Control Protocol (LLCP), welches es beiden Kommunikationspartnern erlaubt, eine Datenübertragung zu initiieren.\cite[S91.f, S.97]{Langer}

        \paragraph{Nutzung unter Android}
        NFC wurde unter Android, verteilt über die API Versionen 9, 10 \& 14,  als {\it NfcManager} und {\it NfcAdapter} zur Verfügung gestellt. Im Gegensatz zu Wifi, wo lediglich Verbindungszustände als Broadcasts gemeldet werden, werden bei Nfc die gelesenen Daten als Broadcasts an alle zuhörenden Apps gesendet, wodurch sich eine gesicherte Verbindung nicht gewährleisten lässt.

    \subsection{Kommunikation über USB}
        Im Gegensatz zu den vorhergehenden Übertragungsmedien ist der Universal Serial Bus (USB) eine kabelgebundene Schnittstelle. Aus der maximalen Kabellänge von 5 Metern
        ergibt sich somit auch die maximale Reichweite der Technologie. USB ist primär darauf ausgelegt, (Peripherie-)Geräte an einen Host anzuschließen.
        Der Host stellt hierbei eine minimale Versorgungsspannung auf dem Bus bereit, sodass Geräte sich beim Host registrieren können und eine höhere Stromversorgung anfordern können.
        Weiterhin übernimmt der Host zwangsläufig auch die Kontrolle über die Verbindung, da Geräte zu jedem Zeitpunkt physisch vom Bus entfernt werden können, sodass Hot-Plug-and-Play
        möglich wird. \cite[S.21-24]{Kelm}
        Da USB eine möglichst breite Anzahl an Geräten zu unterstützen versucht, ist es nötig in den meisten Fällen einen Treiber auf dem Host bereitzustellen, sodass Anwendungen mit dem USB-Gerät kommunizieren können.\cite[S.197]{Kelm}

        \paragraph{Nutzung unter Android}
        Durch das Android Open Accessory Framework, welches ab API 10 angeboten wird, ist es möglich, mit einem Android-Gerät über USB zu kommunizieren und dabei als Host zu agieren. Es ist hierbei jedoch nötig einen Treiber auf Host-Seite zu schreiben, um dieses Framework nutzen zu können.\cite{AOA}

    \subsection{Evaluation der Übertragungsmedien}
    Alle vorgestellten Schnittstellen, außer USB, lassen sich unter Android, wie in \cite{test-repository} zu sehen, mit ähnlichem Aufwand anbinden.
    Da für USB ein Treiber auf Hostseite nötig ist, der AOA implementiert, ist der Aufwand wesentlich höher im Vergleich zu den restlichen Schnittstellen.
    Unter Pharo wurde bisher keine dieser Kommunikationskanäle angebunden,
    jedoch lassen sich bereits Sockets in Pharo nutzen und damit MAC-Pakete versenden.
    Da WLAN die größte Reichweite im Vergleich zu den restlichen gezeigten Technologien bietet, stellt sich hierbei ebenfalls Wi-Fi Direct als sinnvollste Schnittstelle heraus.
    Da nicht jedes Android-Gerät Wi-Fi Direct unterstützt, ist es jedoch sinnvoll, die Implementierung der Konfigurationsschnittstelle so zu kapseln,
    dass sie auch über andere Kanäle angesprochen werden kann.
    
    Der Austausch über Bluetooth über verfügbare Dienste der Geräte ist erst nach einer vollständigen Kopplung und Verbindung möglich.
    Dadurch ist ein Filtern der verfügbaren Bluetooth Geräte zur Auswahl durch den Nutzer nur über den Namen oder die Adresse des Gerätes möglich.
    Im Gegensatz dazu ermöglicht Bluetooth Low Energy eben genau die gleiche Funktionalität wie Wi-Fi Direct, indem Daten ohne eine aktive p2p-Verbindung übertragen werden können. Es sollte somit Bluetooth nur in Kombination mit Bluetooth Low Energy genutzt werden.

    Um dem Nutzer möglichst einfach erreichbare Geräte zeigen zu können, ist es wünschenswert, dem Nutzer wenig bis keinen Aufwand bei der Anbindung des Gerätes über die Übertragungsmedien zu geben.
    Wi-Fi Direct bietet hierbei mit DNS SD eine Lösung, um Dienste ohne Konfiguration auf Seite von Client-Geräten vorzustellen, die keinerlei Aufwand für den Nutzer bedeutet.
    Bluetooth Low Energy bietet ebenso die Möglichkeit, Beacon-Signale mit Nutzdaten zu versenden, wodurch die IoT-Geräte auf Client-Geräten ohne Nutzereingabe identifiziert werden können.
    Für NFC und USB sind ein physischer Zugang zu dem IoT-Gerät notwendig, wodurch sie in manchen Szenarien, wie die Montage an einer Decke, einen relativ hohen Aufwand für den Nutzer bedeuten.
    NFC ermöglicht eine Datenübertragung nur für einen kurzen Zeitraum, wodurch es für dieses Projekt weniger praktisch als eine USB-Verbindung ist.
    Da das Ergebnis der WLAN-Verbindungsversuche auf dem Smartphone zu sehen sein soll, würde somit ein mehrfacher Verbindungsaufbau über NFC nötig werden.

    \pagebreak
    \section{Architektur}
    Unabhängig des verwendeten p2p Übertragungsmediums soll eine Schnittstelle in Form eines auf Requests antwortenden Servers zur Verfügung stehen.

    und sich eine REST Schnittstelle durch die Pakete ...\footnote{TODO} leicht in pharo definieren lässt, bietet es sich an, diese fertigen Serverkomponenten zu nutzen.
    Um REST als Schnittstelle jedoch nutzen zu können muss der genutzte Kommunikationskanal HTTP übertragen können. Da jede der vorgestellten Schnittstellen Bytes oder Strings
    übertragen kann, stellen die Übertragungsmedien kein Problem dar, jedoch muss der REST-Server im Falle von Bluetooth, NFC und USB die HTTP-String Repräsentationen zur Verfügung stellen können,
    ohne sie über eine TCP/IP-Netzwerkschnittstelle zu versenden. Ebenso muss der Client die HTTP-Nachrichten korrekt verarbeiten können.
    Da die meisten Bibliotheken an den TCP/IP-Stack von Android gekoppelt sind, um einen gekapselten HTTP-Client anbieten zu können, ist es nicht immer möglich eine String-Repräsentation des zu tätigenden Aufrufs zu erhalten oder an die Bibliothek zu übergeben.
    Die OkHttp Bibliothek sollte diese Probleme jedoch umgehen können, da Netzwerk Aufrufe durch eine Kette von Interceptoren gereicht werden und der RealNetworkInterceptor,
    welcher sich als letztes in dieser Kette befindet, überschrieben werden kann.\footnote{TODO: gibt OkHttp fertige Http Aufrufe in den letzten Interceptor?}
	Im folgenden wird jedoch nur noch WiFi Direct und damit eine TCP/IP-Implementierung betrachtet.
    Um die Schnittstelle einfach warten zu können, wird sie als JSONSchema über OpenApi dokumentiert und dem Client zur Verfügung gestellt.\footnote{TODO: JSON SCHEMA und OpenApi sollten erklärt werden}

    Die Schnittstelle muss folgende Funktionalitäten anbieten:
    \begin{itemize}
        \item Eine Auflistung der verfügbaren Netzwerkschnittstellen sollte ähnlich zu {\it ip link show} zur Verfügung stehen, um die ID der gewünschten Schnittstelle für die nächsten Aufrufe herauszufinden.
        \item Pro Netzwerkschnittstelle soll es möglich sein, sich den aktuellen Verbindungsstatus sowie die akutelle Konfiguration ausgeben sowie anpassen zu lassen.
        \item Eine Ausgabe von Systemlogs über die Netzwerkschnittstellen soll ungefiltert und pro Schnittstelle gefiltert zur Verfügung stehen.
        \item Die Spezifikation der Schnittstelle soll als Aufruf der Schnittstelle zur Verfügung stehen, sodass sie leicht nutzbar ist.
    \end{itemize}

    \includegraphics{IOT-Connectivity-Protocol-Stack}

    REFERENCE TO IMAGE zeigt, welche Elemente sich im Protokoll-Stack befinden, um die Serveranwendung als REST-Server Clientgeräten zur Verfügung zu stellen.
    Der REST-Server lässt sich hierbei als eine OpenApi/JsonSchema-Implemnetierung der Zweidenker GmbH nutzen.\footnote{https://github.com/zweidenker/OpenAPI}

    
	\begin{figure}[ht]
		\centering
	    \includegraphics[width=0.85\textwidth]{../latex-ai-project/IOT-Connectivity-Protocol-Stack}
    	\caption[]{Zu sehen sind die Schnittstellen, die nötig sind, um die Client-Server Anwendung über p2p nutzen zu können. Hierbei stehen alle rot hinterlegten Schnittstellen bereits zur Verfügung. Wenn die genutzte p2p Schnittstelle es nicht erlaubt, TCP/HTTP Pakete an die Ports einer Netzwerkschnittstelle des Servers zu senden, muss ein TCP/HTTP-Wrapper erstellt werden. Andernfalls besteht die Implementierung aus einem gewöhnlichem REST-Client/Server-Paar. Getrennt davon muss jedoch auch innerhalb der Anwendung die p2p-Service-Discovery angebunden werden. }
	    \label{protocol_stack}
	\end{figure}     
    
    \section{Umsetzung}
Um eine stabile Lösung schnell und simpel anbieten zu können, wird zur Implementierung der REST-Schnittstelle auf bestehende Server sowie Client Bibliotheken zurückgegriffen. Um diese Schnittstelle mit Funktionalität füllen zu können, 
\subsection{Pharo}
\subsection{Android}

\subsection{Stolpersteine}
    
    \begin{figure}[ht]
		\centering
	    \includegraphics[width=1.0\textwidth]{../latex-ai-project/user_flow.png}
    	\caption[]{Der Benutzer sieht zunächst eine Liste an konfigurierbaren Geräten in der Nähe (links oben), durch Auswahl bekommt der Nutzer die Liste der verfügbaren Netzwerkschnittstellen zu sehen (links unten). Sobald auch hier eine Auswahl erfolgt ist, kann der Nutzer eine Liste an verfügbaren und bereits konfigurierten Netzwerken für diesen Netzwerkadapter sehen (mitte). Letztlich kann der Nutzer noch ein Netzwerk durch Eingabe des Passworts konfigurieren (rechts). }
	    \label{user_flow}
	\end{figure}
	
	\section{Ausblick}
	Im Rahmen dieses Projektes wurde Wi-Fi Direct zur Nutzung als p2p Technologie zwischen Android und pharo untersucht. Zwecks mangelnder Dokumentation von Wi-Fi Direct und dessen Implementierung unter Linux war es nicht möglich, eine stabile Verbindung wiederholbar aufzubauen. Da Service Discovery jedoch größtenteils stabil über Wi-Fi Direct funktioniert, ist zu überlegen, ob eine Hybrid Lösung in Verbindung mit Bluetooth oder anderen p2p Technologien sinnvoll ist. Damit wäre es möglich, manche Nachteile von anderen Technologien auszugleichen. In Kombination mit Bluetooth könnte so die Broadcastreichweite spürbar erhöht werden. Generell sollten somit andere Technologien auch auf deren Nutzbarkeit überprüft werden. Für Diese wird es dann auch nötig, einen Wrapper um Socketverbindungen zu bauen.
	Ebenso ist es denkbar, mehrere p2p Technologien parallel zu betreiben, damit der Nutzer nicht durch eine fehlende Technologie in seinem Smartphone eingeschränkt wird. Hierbei sollte auch darauf geachtet werden, dass der Nutzer nicht von dieser Parallelität behelligt wird, somit die Auswahl automatisch geschieht.
	
	Der implementierte REST-Server verwendet noch den standardmäßigen Error Handler, der im Fehlerfall einen Stacktrace als plain text zurückliefert. Dies sollte abgefangen werden und in ein JSON Objekt mit Fehlercodes und sprechenden Fehlermeldungen umgewandelt werden. Aktuell führt der REST-Server noch alle Aufrufe ohne Authentifizierung durch. Es sollte jedoch wenigstens eine Authentifizierung bei der Erstkonfiguration festgelegt werden, sodass nicht autorisierte Personen keine Änderungen an den Einstellungen vornehmen können.
	In Verbindung mit dieser Authentifizierung auf Anwendungsebene kann auch das Gerät an ein virtuelles Netzwerk gebunden werden, welches eigene Geräte bündelt. Dieses virtuelle Netzwerk wird dazu genutzt, dass die Geräte einen Server in diesem Netzwerk besitzen, auf den sie sich beziehen können. Dies kann entweder eine lokal laufende pharo Instanz sein oder ein Cloudservice, an dem das virtuelle Netzwerk einem Account entspricht. Damit wird nicht nur der Zugriff lokal sondern auch Remote eingeschränkt. Um diese virtuellen Netzwerke gut verwalten zu können, empfiehlt es sich auch hierbei ein DNS Service Discovery einzusetzen, die Daten jedoch durch verschiedene Maßnahmen zu schützen.\cite[S.8]{AI-Kaiser2}
	
	Im Rahmen von Low Energy Netzwerken könnten die Broadcasts der Service Discovery von anderen pharo Instanzen auch erkannt werden und so gebündelt werden. Dies erlaubt es, Energie dadurch einzusparen, dass nur noch ein Gerät aus dem Low Energy Netzwerk den Broadcastanfragen antworten muss. Ebenfalls lässt sich so die Reichweite von p2p Verbindungen erhöhen, da die Anfragen zum entsprechenden Ziel weitergeleitet werden können. Im Hinblick auf Bandbreite stellt dies erst ein Problem dar, wenn mehrere Geräte im selben Netzwerk parallel administriert werden, was jedoch einen Randfall darstellen sollte und somit vorerst ignoriert werden kann. Es ist somit also denkbar die pharo Instanzen als Knoten in einem Netzwerk aus oben genannten Gründen agieren zu lassen. Diese Knoten müssten dann jedoch ebenfalls Routingtabellen verwalten, was dieses dezentrales Netzwerk im Vergleich zum Nutzen, der daraus gewonnen wird, zu kompliziert macht.
	
	

	
	\pagebreak
    \renewcommand\refname{Quellen}
    \begin{thebibliography}{10}
        \bibitem[Aneja et al.]{AI-Aneja}Nagender Aneja und Sapna Gambhir: "Profile-Based Ad Hoc Social Networking Using Wi-Fi Direct on the Top of Android" {\it Mobile Information Systems, Volume 2018, Article ID 9469536, 7 pages} (2018).
        \bibitem[Eberhardt et al.]{AI-Kelm} Udo Eberhardt und Hans Joachim Kelm [Hrsg.]: {\it USB - Universal Serial Bus} Franzis, Poing (1999).
        \bibitem[Esnaashari et al.]{AI-Esnaashari} Shadi Esnaashari, Ian Welch und Peter Komisarczuk: {\it Determining home users' vulnerability to Universal Plug and Play (UPnP) attacks} Erschienen in: Proceedings of the 2013 27th International Conference on Advanced Information Networking and Applications Workshops (WAINA). IEEE, 2013. - S. 725-729. - ISBN 9781467362399 .
        \bibitem[Kaiser et al.]{AI-Kaiser} Daniel Kaiser, Marcel Waldvogel, Holger Strittmatter und Oliver Haese: {\it User-Friendly, Versatile, and Efficient Multi-Link DNS Service Discovery} Erschienen in: Proceedings 2016 IEEE 36th International Conference on Distributed Computing Systems Workshops : ICDCSW 2016. - Piscataway, NJ : IEEE, 2016. - S. 146-155. - ISBN 978-1-5090-3686-8.
        \bibitem[Kaiser, Waldvogel]{AI-Kaiser2} Daniel Kaiser und Marcel Waldvogel: {\it Adding Orivacy to Multicast DNS Service Discovery} Erschienen in: Proceedings - 2014 IEEE 13th International Conference on Trust, Security and Privacy in Computing and Communications (TrustCom) ; Beijing, China, 24 Sep - 26 Sep 2014. - Piscataway : IEEE, 2014. - S. 809-816. - ISBN 978-1-4799-6513-7.
        \bibitem[Langer et al.]{AI-Langer}Josef Langer und Michael Roland: {\it Anwendungen und Technik von Near Field Communication (NFC)} Springer, Heidelberg (2010).
        \bibitem[Lüders]{AI-Lueders}Christian Lüders: {\it Lokale Funknetze: Wireless LANs (IEEE 802.11), Bluetooth, DECT} Vogel, Würzburg (2007).
        \bibitem[Morrow]{AI-Morrow}Robert Morrow: {\it Bluetooth Operation and Use} McGraw-Hill (2002).
        \bibitem[Sauter]{AI-Sauter}Martin Sauter: {\it Grundkurs Mobile Kommunikationssysteme: LTE-Advanced, UMTS, HSPA, GSM, GPRS, Wireless LAN und Bluetooth} 6.Auflage Springer Vieweg, Wiesbaden (2015).
        \bibitem[Sikora]{AI-Sikora}Axel Sikora: {\it Wireless LAN: Protokolle und Anwendungen} Addison-Wesley, München u.A. (2001).
		\bibitem[Townsend et al.]{AI-Townsend} Kevin Townsend, Carles Cuffí, Akiba und Robert Davidson: {\it Getting Started with Bluetooth Low Energy} O'Reilly (2014).
		
		\bibitem[Android Koin]{AI-androidKoin}https://insert-koin.io/
		\bibitem[Android Koin Speed]{AI-androidKoinSpeed}https://medium.com/koin-developers/ready-for-koin-2-0-2722ab59cac3
        \bibitem[Android Accessory]{AI-AOA}https://source.android.com/devices/accessories/custom
        \bibitem[Android OkHttp]{AI-AndroidOkHttp}https://github.com/square/okhttp
        \bibitem[Android Permissions]{AI-androidRights}https://elinux.org/Android\_Security\#Paranoid\_network-ing
        \bibitem[Android Repository]{AI-androidRepo}https://android.googlesource.com/platform/frameworks/base
        \bibitem[iot-connectivity]{AI-main-repository}https://github.com/janphkre/iot-connectivity
        \bibitem[iot-p2p-test]{AI-test-repository}https://github.com/janphkre/iot-p2p-test
        \bibitem[JSON Schema]{AI-JsonSchema}https://json-schema.org/
        \bibitem[Linux Packet]{AI-linuxPacket}http://man7.org/linux/man-pages/man7/packet.7.html
        \bibitem[OpenAPI]{AI-OpenApi}https://swagger.io/docs/specification/about/
        \bibitem[Pharo]{AI-pharo}http://pharo.org/
        \bibitem[Pharo OpenAPI]{AI-pharoOpenApi}https://github.com/zweidenker/openapi
        \bibitem[Pharo Things]{AI-pharoThings}https://github.com/pharo-iot/PharoThings
        \bibitem[Pharo Zinc]{AI-pharoZinc}https://github.com/zweidenker/zinc/
        \bibitem[WiFi Direct]{AI-wifiDirect}https://www.wi-fi.org/discover-wi-fi/wi-fi-direct
        \bibitem[WiFi HostAP Repository]{AI-hostAp}git://w1.fi/hostap.git
        \bibitem[WiFi wpa\_supplicant]{AI-wpaSupplicant}https://w1.fi/wpa\_supplicant/
        \bibitem[Zweidenker]{AI-zweidenker}https://zweidenker.de
    \end{thebibliography}
    \end{leveldown}
    \stopcontents[sections]

    \begin{titlepage}
    \section*{Eidesstattliche Erklärung}
    Ich versichere hiermit, die vorgelegte Arbeit in dem gemeldeten Zeitraum ohne fremde Hilfe verfasst und mich keiner anderen als der angegebenen Hilfsmittel und Quellen bedient zu haben.
    ~\\[1.0cm]
    \begin{flushright}
    Wermelskirchen, den 23. August 2019
    ~\\[1.0cm]
    \rule{8cm}{1pt}\linebreak
    Jan Phillip Kretzschmar
    \end{flushright}
    \end{titlepage}
\end{document}
