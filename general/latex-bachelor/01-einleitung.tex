\section{Einleitung}
        In der Firma {\it Zweidenker GmbH} soll ein Internet of Things System auf Grundlage von PharoThings \cite{pharoThings} geschaffen werden, um Sensoren und Aktoren in einem Netzwerk oder zu einem Server verfügbar zu machen.
        Im Internet of Things (IoT) werden Geräte miteinander vernetzt, die nicht zwingend eine Internetanbindung benötigen würden, jedoch erweiterte Funktionalitäten der Geräte dem Nutzer so zur Verfügung gestellt werden. Um solchen Geräten eine Internetanbindung geben zu können, ist es nötig, eine kabelgebundene Verbindung zu schaffen oder die Geräte in ein kabelloses Wi-Fi Netzwerk aufzunehmen. Die Verbindungskonfiguration von IoT Geräten für Wi-Fi Netzwerke wurde bereits auf Basis von Wi-Fi Direct untersucht \cite{aiProject}. Da die Implementierung dieser Technologie einige Unzulänglichkeiten besitzt, soll nun ein Vergleich von weiteren p2p Technologien \linebreak vorgenommen werden. Hierunter fallen Bluetooth, Near Field Communication (NFC) und Universal Serial Bus (USB), da diese alle von Android Smartphones nativ unterstützt werden. Da die Vor- und Nachteile sowie Funktionsweisen dieser p2p Schnittstellen bereits bekannt sind, sollen sie im Rahmen dieser Arbeit in die \linebreak Verbindungskonfiguration mit PharoThings eingebunden werden, um auch hier Probleme aufzudecken und zu einer optimalen Lösung des Problems zu gelangen.
        
        Die existierende Lösung besteht aus einer Service Discovery und einer p2p \linebreak Verbindung auf Basis von Wi-Fi Direct. Die Service Discovery wird als bestehend beibehalten, um die Evaluation auf die Nutzbarkeit der p2p Verbindung zu konzentrieren. Ziel soll sein, eine Alternative p2p Technologie zu Wi-Fi Direct zu finden.

        \subsection{Zielsetzung}
        Im vorausgegangenen Praxisprojekt wurde deutlich, dass Wi-Fi Direct einige \linebreak gravierende Mängel im Hinblick auf die Funktionalität und Robustheit der bereitgestellten Bibliotheken aufweist. So ist die Dokumentation der meisten Nachrichten und Events unvollständig oder fehlt, zudem ist unklar wie eine Verbindung aufgebaut werden muss. Gleichzeitig sind Fehlermeldungen entweder nicht aussagekräftig oder führen dazu, dass der laufende Daemon abstürzt.
        Daraus lassen sich für diese Evaluation zwei Kriterien im Bereich der Robustheit ableiten.
        Zunächst kann getestet werden wie aussagekräftig Fehlercodes oder Fehlermeldungen sind.
        Das eine Ende des Spektrums bildet hierbei die Darstellung des Erfolges eines Aufrufs durch eine boolesche Variable, auf der anderen Seite wird ein genauer Fehlercode in Verbindung mit einer Fehlermeldung ausgegeben. Die Fehlercodes sind dabei für jede genutzte Methode dokumentiert und bieten Aufschluss auf die Gründe der erhaltenen Fehlschläge. 
        Gleichzeitig soll evaluiert werden, inwieweit sich von aufgetretenen Fehlern erholt werden kann, um dennoch eine erfolgreiche Verbindung aufbauen zu können. Hierbei wird der messbare Bereich auf der einen Seite durch den Absturz des verwendeten Moduls beschränkt. Es besteht somit keine Möglichkeit zur Erholung von Fehlern und es muss darüber hinaus sogar das verwendete Modul neu gestartet werden. Dem gegenüber steht die gute Dokumentation von Fehlercodes, welche es erlaubt, auf nicht kritische Fehler reagieren zu können. Ebenso können die Fehler bereits einen Codeblock beinhalten, der es erlaubt, den Fehler, so fern dieser unerwünscht gewesen ist, zu beseitigen.
        Ebenso sollte die Anbindung einer eingesetzten Technologie testbar sein. Die Testbarkeit der Software ergibt sich zum einen aus einem hohen Abstraktionsgrad, so dass sich die genutzten Technologien mit wenig Aufwand durch Mocks und Stubs auswechseln lassen. Ebenso ist eine Testbarkeit erst dann gegeben, wenn die genutzt Technologie eine hohe Stabilität im Hinblick auf Wiederholbarkeit bietet.
        
        Die Funktionalität von Wi-Fi Direct wird aktuell durch mehrere Faktoren \linebreak eingeschränkt. Zunächst mangelt es an einem internen Zustandsdiagramm des genutzten Moduls. Dadurch ist es nicht möglich, festzustellen, welche Methoden in einer bestimmten Reihenfolge aufgerufen werden müssen, um den erwünschten Zu-stand einer Verbindung zu erreichen. Zudem sind hierbei interne Zustandsübergänge durch ankommende Events oder weitere unbekannte Gründe nicht dokumentiert. Dies führt dazu, dass nicht klar ist, wann Methoden aufgerufen werden müssen um den aktuellen Zustand der p2p Schnittstelle beizubehalten.
        Als Evaluations-kriterium kann hierbei wieder die Dokumentation im Hinblick auf die \linebreak Funktionalität dienen, denn es kann festgestellt werden, wie gut der Verbindungszustand und Modulzustand dokumentiert sind. Es besteht zum Einen die Möglichkeit, dass keine Dokumentation vorliegt und das genutzte Modul lediglich als Blackbox genutzt werden kann. Als Optimum sollte jedoch die Dokumentation soweit vorhanden sein, dass interne sowie externe Events dokumentiert sind und gemeinsam mit einem Zustandsdiagramm sowohl der Verbindung als auch der Software dazu genutzt werden können, genau nachzuvollziehen, wann sich der aktuelle Zustand ändert und wie dieser wiederhergestellt werden kann.
        
        Die Qualität der eingesetzten Technologie kann zudem auf Effizienz überprüft werden. Da die genutzten Module in das Gerät fest integriert sind, lässt sich Energieverbrauch nicht messen und ist außerdem von dem im Gerät konkret verbauten Chip und dessen Treibern abhängig. Es kann jedoch eine Aussage über Übertragungsraten getroffen werden, welche eine numerische Skala abbilden. Hierdurch kann festgestellt werden, inwieweit manche Anwendungsfälle mit der entsprechenden Technologie möglich sind, oder die Nutzbarkeit dadurch eingeschränkt werden. Die Konfiguration über eine REST-Schnittstelle benötigt keine hohen Datenraten zur Kommunikation, falls der Nutzer jedoch beispielsweise eine Remote-Verbindung durch Telepharo über die p2p Schnittstelle zwischen seinem persönlichen Computer und dem eingesetzten IoT-Gerät aufbauen will, benötigt er eine relativ latenzfreie und großvolumige Datenübertragung.
        
        Letztlich kann überprüft werden, wie leicht sich die Technologie implementieren lässt, sodass sie für das bestehende Projekt genutzt werden kann. Da die Schwierigkeit von Aufgaben im Rahmen von Softwareentwicklung immer eine Kombination aus Zeitaufwand und Komplexität sind, lässt sich hierbei immer nur eine Schätzung oder persönliche Meinung auf Basis der eigenen Präferenz angeben.
    \subsection{Evaluationsmethodik}
    	Um die genannten Evaluationsziele überprüfen zu können, soll jede der Technologien Bluetooth, NFC und USB in einem vergleichbaren Maße zu Wi-Fi Direct im Hinblick auf die Vollständigkeit ihrer Dokumentation und Nutzbarkeit ihrer Implementierung überprüft werden.
    	Unter dem Aspekt der Softwarequalität lassen sich die Technologien gegenüberstellen. Dabei soll auf Grundlage einer beispielhaften Implementierung im Rahmen des bestehenden Projektes die Vor- und Nachteile verdeutlicht werden und zudem eine Evaluation der Komplexität und des Aufwandes einer Implementierung vorgenommen werden. Metriken, um die Qualität der Software zu messen, sind als ISO-Standard 9126 definiert \cite[S.6]{liggesmeyer}. Hierbei soll auf die folgenden Punkte eingegangen werden und wie diese extern gemessen werden können, da nicht einsehbar ist, wie ausgelieferte Implementierungsartefakte intern getestet werden und dokumentiert sind. Es soll so auf die Punkte Zuverlässigkeit, Wartbarkeit (Instandhaltbarkeit), Funktionalität, Effizienz und Benutzbarkeit als Softwarequalitätsmerkmale eingegangen werden.
    	\begin{enumerate}
    	\item {\it Zuverlässigkeit:} 
    	Um die Robustheit und Zuverlässigkeit der Technologien feststellen und vergleichen zu können, muss festgestellt werden können, wie viele Fehler in der genutzten Implementierung existieren \cite[S.9]{liggesmeyer}. Da dieser Punkt jedoch unter anderem stark von der genutzten Hardware abhängt soll sich zunächst auf das Android Gerät Samsung Galaxy S7 und einen Raspberry Pi 3 B+ beschränkt werden. Die Anzahl von Fehlern in einer Technologie lässt sich ebenfalls nicht messbar festsetzen, da zwar die Zahl von gemeldeten Fehlern in einer Anwendung über öffentliche Bugtracker der eingesetzten Bibliothek nachvollziehbar ist, jedoch so nicht alle Fehler abgefangen werden können, da zum einen auch Fehler in Treibern und Hardwaredesign bestehen können und ebenso ein Bugtracker nicht zwingend die vollständige Menge aller Fehler enthalten kann. Gleichzeitig sollte hierbei zur Evaluation die Qualität der Dokumentation bewertet werden. Dazu zählt die Dokumentation und Verfügbarkeit des Sourcecodes im Hinblick auf die Schicht, welche als Schnittstelle bereit gestellt wird.
    	\item {\it Wartbarkeit:} Um Fehler bei der Benutzung einer der Technologien beheben zu können, ist es nötig, dass die Technologie auf der einen Seite durch Dokumentation und Quellcode leicht zu verstehen ist und gleichzeitig es erlaubt, reproduzierbar Fehler zu testen. Um hierbei eine aussagekräftige Bewertung vornehmen zu können, sollte keine Metrik wie die Häufigkeit von abweichenden Ergebnissen bei gleichen Eingabeparametern genutzt werden, da dies von nicht überschaubaren Faktoren abhängt. Es kann daher nur auf den Abstraktionsgrad der angebotenen Schnittstellen eingegangen werden, da diese für eigene Tests ersetzt werden müssen. Jedoch kann dieser Punkt auch in Verbindung mit der Zuverlässigkeit gesehen werden, da Fehler bei einer sauberen Dokumentation kein großes Hindernis für Wiederholbarkeit und damit Testbarkeit darstellen.
    	\item {\it Funktionalität:}
    	Die Abdeckung der Implementierungen nach der Definition einer Vollständigkeit der erwarteten Funktionalität \cite[S.8]{liggesmeyer} kann nur sehr eingeschränkt überprüft werden, da aus einer externen Sicht die Menge der Funktionalitäten sich auf den Verbindungsaufbau und Verbindungsabbau, sowie ein Senden und Empfangen von verbindungslosen Daten beschränkt und alle Technologien diese Funktionalitäten anbieten. Es muss somit darauf geachtet werden inwieweit die internen Zustände dokumentiert sind und erreicht werden können.
    	\item {\it Effizienz:}
    	Die Effizienz der Technologien kann durch einen Vergleich von Datendurchsatzraten und Antwortzeiten festgestellt werden. Diese beiden Kennzahlen sind für eine Verbindungskonfiguration nicht kritisch, jedoch können mittels eines externen HTTP-Wrappers auch Drittprogramme eine solche p2p Verbindung nutzen und in Anwendungsfällen wie Streaming von Audio und Video werden solche Raten dann interessant. Eine andere Seite der Effizienz im Hinblick auf Energieverbrauch der Technologien wird vor allem im Betrieb eines Gerätes mit einer Batterie wichtig. Ein Stromverbrauch lässt sich jedoch ebenfalls kaum bewerten, da dies stark von der verbauten Hardware abhängt. Es ist auch für eine Verbindungskonfiguration zunächst unerheblich, somit wird auf eine Bewertung der Energieaufnahme verzichtet.
    	\item {\it Benutzbarkeit:}
    	Die Benutzbarkeit der eingesetzten p2p Lösung, ergibt sich zum Einen daraus, wie stark der Nutzer in die verwendete Technologie eingebunden werden muss und zum Anderen daraus, wie komplex und zeitaufwändig eine Anbindung der Technologie ist.
    	Der Nutzer ist zum größten Teil dazu angehalten, einen Verbindungsaufbau als erwünscht zu bestätigen. Dieser Vorgang sollte so leicht wie möglich gestaltet sein und kann durch eine \linebreak Gegenüberstellung der einzelnen Schritte, die der Nutzer ausführen muss, bis eine erfolgreiche Verbindung besteht, verglichen werden.
    	Die Komplexität und der Aufwand einer Implementierung kann anhand der Menge an Methoden, die implementiert werden müssen, besonders in einer C-Bibliothek, die eine eigene \linebreak Schnittstelle zu pharo abbildet, festgestellt werden. Außerdem ist ein Vergleich im Hinblick auf die Komplexität der Logik, die bei der Anbindung gehandhabt werden muss, nötig.
    	\end{enumerate}
	    	\pagebreak
    		\subsection{Messbarkeit}
		Da Softwarequalität ebenso wie die benannten Metriken größtenteils nicht quantitative messbare Kriterien sind, können auch keine Referenzwerte für die einzelnen Kriterien zu Hilfe gezogen werden. Eine quantitative Aussage wie eine Anzahl an exi-stierenden Tests ist ebenfalls nur bedingt aussagekräftig, da nicht bewertet werden kann, wie robust ein Test geschrieben wurde und welche Ergebnisse getestet werden. Der letztere Punkt lässt sich teilweise über eine Analyse der Quellcodeabdeckung der Tests erreichen, jedoch entspricht eine Implementierung einer Technologie nicht zwingend ihrer Spezifikation und erlaubt so dennoch Abweichungen in den getesteten Ergebnissen und somit auch Inkompatibilitäten zu anderen Implementierungen der selben Technologie.
		
		Eine Bewertung kann somit primär nur auf Grundlage der gesammelten Erfahrungswerte und der vorgenommenen Umsetzung mit den zu überprüfenden Technologien getroffen werden. 
		
		\subsection{Einschränkungen}
		Um den Aufwand der Arbeit überschaubar zu halten, gibt es ein paar weitere Rahmenbedingungen, die im Vorfeld definiert werden sollen. Hierbei ist darauf zu achten, wenn bei einer der Technologien ein deutlich größerer Aufwand abzusehen sein sollte, muss die Implementierung dieser Technologie ausgeklammert werden, da es auch einer Nutzbarkeit widerspricht, eine solche Technologie trotz einem hohen Aufwand in der Implementierung zu nutzen, da dies gleichzeitig auch einen hohen Aufwand in der Wartung bedeutet.
		
		Ein Bezug zwischen Kosten und Nutzen lässt sich über ein Qualitätsmodell ebenfalls nur schlecht aufstellen, da alle Technologien frei zugängliche Software besitzen und Kostenunterschiede lediglich durch die genutzte Hardware entstehen. Bei NFC entstehen als Einziges Mehrkosten, da es nicht bereits im Raspberry Pi integriert ist. Es wird somit darauf verzichtet, die Nutzbarkeit der einzelnen Technologien anhand ihrer Kosten gegenüberzustellen.
		
		Weiterhin kann nicht auf Dokumentationen der entsprechenden Standards und Implementierungen der Technologien eingegangen werden, die nicht öffentlich \linebreak zugänglich sind, da ebenso Änderungen und Erweiterungen an diesen Spezifikationen dann nicht nachvollziehbar sein werden und ein Implementierungsaufwand einer eigenen Anbindung somit unvergleichbar hoch ist.
		
		
