\section{Fazit}

Die Überprüfung von Softwarequalität auf einer Kombination aus Technologie und Software stellt die Herausforderung, bewertbare Ergebnisse zu erhalten. Da die Qualitätsmerkmale der Zuverlässigkeit, Wartbarkeit, Funktionalität und Bedienbarkeit lediglich auf Grund von Erfahrungswerten bewertet wurden und ihre Spezifikation wesentlich zu komplex sind, um sie im Rahmen dieser Arbeit vollständig auszuwerten, kann nur eine informale Aussage über die eingesetzten Technologien getroffen werden. Jede der Technologien setzt unterschiedliche Schwerpunkte bei den überprüften Qualitätskriterien, wodurch jede der Technologien für manche Anwendungsfälle in Frage kommt. Der Anwendungsfall einer Verbindungskonfiguration als Client-/Server-Lösung zeichnet sich durch ein geringes Datenvolumen der Anfragen und Antworten in Verbindung mit teilweise hohen Antwortzeiten auf Anwendungsebene aus. Auf Grund dieser Gegebenheiten wird der Evaluationspunkt der Effizienz übergangen und lediglich anhand der verbleibenden Evaluationspunkte ein Ergebnis der Evaluation genutzt.

Obwohl sich die drei überprüften Technologien in ihrer Funktionsweise grundlegend unterscheiden, können sie alle für eine p2p Verbindung zwischen zwei Endnutzergeräten verwendet werden. Die eigentliche Art und Weise der angebotenen Funktionalität rückt bei der Umsetzung deutlich in den Hintergrund, da jede genutzte Bibliothek eine API bereitstellt, über die eine vereinfachte Steuerung der Verbindung zu anderen Geräten ermöglicht wird.

Für jede der genutzten Technologien kommt ein HTTP Wrapper zum Einsatz, welcher es ermöglicht, die HTTP Befehle über eine andere Technologie als Wi-Fi zum Server zu übertragen. Dieser Wrapper ist eine simple Lösung, die jegliche Technologie als Sockets abbildet, um eine simple Auswechselbarkeit der Technologien zu ermöglichen. Dies hilft ebenfalls das gesetzte Ziel einer Alternative zu Wi-Fi Direct zu erfüllen, da sich die unterschiedlichen Technologien mit geringem Aufwand auch parallel in der Anwendung nutzen lassen und so dem Nutzer eine Wahl zwischen mehreren Verbindungsarten gelassen werden kann. Ebenso müssen keine Änderungen an den Anbindungen der Technologien vorgenommen werden, wenn die Schnittstelle zur Verbindungskonfiguration erweitert werden soll. Da alle Anbindungen von Technologien die selbe Grundlage nutzen, um sie im Wrapper nutzen zu können, entsteht kein zusätzlicher Wartungsaufwand der existierenden Anbindungen, wenn eine neue Technologie ebenfalls genutzt werden soll.

Wichtig ist auch, dass die definierte REST-Schnittstelle in Verbindung mit dem HTTP Wrapper unverändert weitergenutzt werden kann. Ein Weiterbestehen der bereits vorhandenen Architektur ist nötig, um eine wartbare Implementierung der verschiedenen Technologien vornehmen zu können, ohne tiefgreifende Änderungen an der Client/Server Architektur vornehmen zu müssen. 

Während der Umsetzung aller p2p Technologien wurde deutlich, dass nicht nur Wi-Fi Direct Unzulänglichkeiten in seiner Implementierung besitzt. Alle getesteten Technologien weisen in unterschiedlichen Punkten Probleme auf, die jedoch nicht zwingend die Nutzbarkeit für eine Verbindungskonfiguration beeinträchtigen. Die aufgetretenen Probleme verdeutlichen jedoch noch einmal die unterschiedlichen \linebreak Ansätze der Technologien.

Bluetooth mangelt es auf Grund des Wechsels der Art von Bibliotheks-API an Dokumentation und allgemeinen Beispielen über die Nutzung der Bibliothek. Gleichzeitig traten Probleme bei der Nutzung von {\it SDP} auf, wodurch eine vereinfachte Lösung über einen expliziten RFCOMM Port genutzt werden muss. Dies hat jedoch den Vorteil, dass die Anbindung von Bluetooth wesentlich simpler ist, da keine DBus-API angesprochen werden muss.

Für eine Nutzung von NFC zeigt sich eine Unzulänglichkeit im Bereich der Benutzbarkeit, da der Nutzer gezwungen ist, das Gerät mehrmals in die direkte Nähe des zu konfigurierenden Gerätes bringen, was je nach Einsatzort keine akzeptable Lösung darstellt. Ebenso basiert NFC auf dem Austausch von kurzen Nachrichten über ein Trägersignal und unterstützt so lediglich eine indirekte Kommunikation  beider Teilnehmer durch das Reagieren auf Befehle, was die Funktionalität der Lösung ein wenig dämpft, da so eine bidirektionale Kommunikation zusätzliche Komplexität birgt.

Eine Verwendung von USB scheitert zur Zeit daran, dass eine AOA Verbindung nicht korrekt aufgebaut werden kann dem kann zu Grunde liegen, dass AOA seit dem Jahr 2012 von Google nicht mehr aktiv weiter entwickelt wurde und inzwischen Fehler in der Anbindung dieses Protokolls existieren. Andererseits ist es auch möglich, dass zusätzlicher undokumentierter Konfigurationsaufwand unter Android vorgenommen werden muss. Da keine erfolgreiche Datenübertragung über USB vollzogen werden konnte, kann diese Technologie auch nicht im Rahmen einer Verbindungskonfiguration genutzt werden. USB besitzt durch die Bindung an ein Kabel gegenüber den drahtlosen Technologien einen erheblichen Nachteil in der Benutzbarkeit, jedoch ist dies auch der Hauptvorteil im Hinblick auf Effizienz. Dadurch, dass Störsignale lediglich stark reduziert auftreten, können wesentlich höhere Datenraten erzielt werden.

NFC ist gegenüber USB und Wi-Fi Direct ebenfalls eine bessere Alternative, um p2p Verbindungen zu ermöglichen. Zwar liegen die Bedienbarkeit und Funktionalität nicht auf dem Niveau von Bluetooth, da NFC eine für den Nutzer und die Implementierung umständlichere Technologie nutzt, jedoch werden diese wieder durch die Wartbarkeit und Zuverlässigkeit des Mediums aufgewogen. Besonders der Punkt der Zuverlässigkeit muss bei dieser Technologie hervorgehoben werden, da hier Reproduzierbarkeit von Ergebnissen durch das Senden von Daten als festgeschriebene Bündel priorisiert wird.

In der Evaluation hat sich so Bluetooth als beste Technologie für eine p2p Verbindung durchgesetzt, weil sie die für eine Verbindungskonfiguration relevanten Kriterien den anderen Technologien gegenüber besser erfüllt. Besonders die Bedienbarkeit durch den Nutzer zusammen mit einer sehr guten Wartbarkeit geben Bluetooth einen großen Vorteil gegenüber den verbleibenden Übertragungsmedien. Bluetooth ist eine stabile und leicht zu nutzende Technologie, die die einfache Benutzbarkeit in den Vordergrund stellt. Die Bedienbarkeit stellt auch das wichtigste Kriterium  der geprüften Softwarequalität dar, da die anderen Merkmal nicht zwingend vom Nutzer wahrgenommen werden und eine Software ohne Nutzer auch keinen Nutzen besitzt.

\subsection{Ausblick}

Im Nachgang kann nun auf Grundlage dieses Prototypen eine Umsetzung der p2p Funktionalität über die einzelnen Technologien stattfinden. Besonders wird hier noch einmal Bluetooth betrachtet werden, da die Bluetooth Lösung über einen festgelegten RFCOMM Port noch mittels des Bluetooth Service Discovery Protocol ersetzt werden sollte. Ebenfalls muss noch Discoverability standardmäßig in der genutzten BlueZ Instanz aktiviert werden.

Da der Fokus dieser Arbeit auf dem Aufbau einer p2p Verbindung lag, kann weiterführend der Fokus auf die Service Discovery geschoben werden. Hier kann besonders Bluetooth Low Energy (LE) betrachtet werden. Es besitzt eine ähnliche Funktionsweise zum bereits implementierten DNS Service Discovery über Wi-Fi Direct jedoch erlaubt es auf eine allgemeinere Art und Weise das verbindungslose Senden von Datenpaketen. Aufgrund der geringen Leistungsaufnahme eignet sich dies auch eher für batteriebetriebene Geräte als Wi-Fi Direct.

Android hat mit der Version Q neue Funktionalität für Wi-Fi Direct und Bluetooth LE bereitgestellt \cite{androidQ}. Es kann hierbei überprüft werden, ob dies lediglich die Nutzung vereinfachen soll oder auch die Stabilität der Wi-Fi Direct Implementierung verbessert wurde.

USB sollte aufgrund einer mangelnden Unterstützung seitens Android in Verbindung mit einer umständlichen Nutzung für ähnlich Anwendungsfälle nicht mehr betrachtet werden. Stattdessen kann sich auf fortschrittlichere p2p Verbindungen konzentriert werden, die kein Kabel zur Datenübertragung mehr benötigen.

Nachdem die Verbindungskonfiguration zufriedenstellend abgehandelt ist, lassen sich für die vorgestellten Verbindungen weitere Anwendungsfälle finden, die von einer p2p Verbindung profitieren. Da REST als Schnittstellentechnologie genutzt wurde, lassen sich neue Anwendungsfälle wie ein p2p Remote Debugging für Pharo leicht umsetzen.