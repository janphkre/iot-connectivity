\documentclass[12pt,a4paper]{article}
\usepackage[utf8]{inputenc}
\usepackage{graphicx}
\usepackage[a4paper, textwidth=15cm, textheight=23cm]{geometry}
\usepackage[section]{placeins}
\usepackage{listings}
\usepackage{titletoc}
\usepackage{tkz-kiviat} 
\usetikzlibrary{arrows}
\usepackage{caption}

% Kotlin template from: https://github.com/cansik/kotlin-latex-listing
\lstdefinelanguage{Kotlin}{
  comment=[l]{//},
  commentstyle={\ttfamily},
  emph={delegate, filter, first, firstOrNull, forEach, lazy, map, mapNotNull, println, return@},
  emphstyle={},
  identifierstyle={},
  keywords={abstract, actual, as, as?, break, by, class, companion, continue, data, do, dynamic, else, enum, expect, false, final, for, fun, get, if, import, in, interface, internal, is, null, object, override, package, private, public, return, set, super, suspend, this, throw, true, try, typealias, val, var, vararg, when, where, while},
  keywordstyle={\bfseries},
  morecomment=[s]{/*}{*/},
  morestring=[b]",
  morestring=[s]{"""*}{*"""},
  ndkeywords={@Deprecated, @JvmField, @JvmName, @JvmOverloads, @JvmStatic, @JvmSynthetic, Array, Byte, Double, Float, Int, Integer, Iterable, Long, Runnable, Short, String},
  ndkeywordstyle={\bfseries},
  sensitive=true,
  stringstyle={\ttfamily},
}

% Smalltalk template from:
\lstdefinelanguage{pharo}{
  commentstyle={\ttfamily},
  alsoletter={:}
  emph={},
  emphstyle={},
  identifierstyle={},
  keywords={self, super, nil, true, false, thisContext},
  keywordstyle={\bfseries},
  morecomment=[s]{"}{"},
  morestring=[b]',
  ndkeywords={new:, at:, put:, yourself, configureDynamic:, p2pServiceAdd:, p2pServiceUpdate, asString, onAnyP2P, as:, with: },
  ndkeywordstyle={\bfseries},
  sensitive=true,
  stringstyle={\ttfamily},
}

%Demotion for appendix from: https://tex.stackexchange.com/questions/61766/demoting-promoting-sections-chapters-etc
\newenvironment{leveldown}% Demote sectional commands
  {\let\section\subsection%
   \let\subsection\subsubsection%
   \let\subsubsection\paragraph%
   \let\paragraph\subparagraph%
   %\let\subparagraph\relax%
  }{}

\renewcommand{\contentsname}{Gliederung}
\renewcommand{\figurename}{Fig.}
\renewcommand{\lstlistingname}{Alg.}
\renewcommand{\lstlistlistingname}{Liste der Algorithmen}
\renewcommand{\listfigurename}{Liste der Diagramme}
\newcommand{\reflst}[1]{\lstlistingname\ \ref{#1}}
\newcommand{\reffig}[1]{\figurename\ \ref{#1}}