\section{Grundlagen der Verbindungstechnologien}
	Da die Nutzbarkeit der Übertragungstechnologien für die Verbindungskonfiguration bereits in \cite{aiProject} überprüft wurden, soll noch einmal jede Technologie in ihrer Funktionsweise kurz vorgestellt werden und der Ablauf eines Verbindungsaufbaus beschrieben werden. Im Gegensatz zu Wi-Fi Direct werden die Zustände beim Verbindungsaufbau intern gehandhabt. Es ist dennoch hilfreich zu verstehen, wie eine Technologie funktioniert, um Fehlerursachen bei einem Verbindungsaufbau leicht identifizieren zu können.
	
	\subsection{Bluetooth}
	Bluetooth besteht aus mehreren Protokollschichten, die für einen Verbindungsaufbau durchlaufen werden. Zunächst wird ein {\it Asynchronous Conneciton-Less} (ACL) Kanal zum gewünschten Gerät geöffnet, über den alle folgenden Kommunikationen stattfinden werden. Dieser Kanal übernimmt die Aufgabe, die Kommunikation mit dem Bluetooth Chip zu steuern und erhaltene Nachrichten an die nächste Schicht zurückzugeben \cite[S.400]{Sauter}. Das {\it Host Controller interface} (HCI) ist nicht Teil des Bluetooth Standards, wird jedoch in den meisten Implementierungen des Bluetoothstacks genutzt, um niedrigere Schichten der Implementierung von höheren Schichten zu trennen. Wie der Name bereits suggeriert, bildet diese Schnittstelle den Übergang zwischen Bluetooth Modul und Treiber \cite[S.65]{miller}. Auf Basis dieser Schnittstelle erhalten die höheren Protokollschichten nur für das Gerät relevante Datenpakete aus dem ACL Kanal.
	
	Das Logical Link Control and Adaption Protocol (L2CAP) wird als erstes Element der höheren Protokollschichten genutzt, um mehrere gleichzeitige Verbindungen unterschiedlicher Anwendungen über einen einzigen Kanal übertragen zu können \cite[S.395]{Sauter}. Das Service Discovery Protocol (SDP) schließt an dieser Stelle an, um entfernten Geräten die bestehenden Services des Gerätes bekannt machen zu können, sodass dieser eine Auswahl des zu verbindenden Services tätigen kann \cite[S.395]{morrow}. RFCOMM erlaubt es eine serielle Schnittstelle über Bluetooth abzubilden. Da es ebenfalls ein fixer Teil des Bluetooth Protokollstacks ist, kann dies von L2DAP auch ohne SDP genutzt werden. Hierbei ist jedoch zu beachten, dass RFCOMM lediglich 30 Ports unterstützt, was 30 gleichzeitigen Anwendungsverbindungen zwischen zwei Bluetooth Geräten entspricht \cite[S.398]{Sauter}. SDP ist nicht in der Lage, diese Anzahl zu erhöhen, da jedoch nicht alle angebotenen Bluetooth Services immer genutzt werden, macht es Sinn, den eigenen Service im SDP zu definieren und so sich von der RFCOMM-Schicht erst einen Port zuweisen zu lassen, wenn die Verbindung tatsächlich genutzt werden soll. Damit eine Anwendung sich so im SDP-Protokoll registrieren kann, muss sie ein entsprechendes Bluetoothprofile, im Falle von RFCOMM das Serial Port Profile, anbieten um die Interoperabilität mit anderen Geräten gewährleisten zu können \cite[S.411]{Sauter}.
	
	\subsection{NFC}
	
	Near Field Communciation baut einen Kommunikationskanal auf Grundlage von Induktion als Trägersignal auf.
	
	\subsection{USB}
	Die Schnittstelle Universal Serial Bus (USB) ist ein kabelgebundenes Bussystem bestehend aus einem Host und vielen Endgeräten mit optionalen Hubs \cite[S.23f]{Kelm}. Ein Hub ist ebenfalls ein Endgerät, er wird jedoch genutzt um über eine USB Schnittstelle mehrere USB Geräte ansteuern zu können.

-AOA: PRINZIPIELL ZWEI BEFEHLE ÜBER USB SENDEN / EMPFANGEN