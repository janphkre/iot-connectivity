\documentclass[12pt,a4paper]{article}
\usepackage[utf8]{inputenc}
\usepackage{graphicx}
\usepackage[a4paper, textwidth=15cm, textheight=23cm]{geometry}
\usepackage[section]{placeins}

\title{Bachelorarbeit: PharoThings-Connectivity SS19}
\begin{document}
	\begin{titlepage}
    \includegraphics[width=0.4\textwidth]{../latex-ai-project/th_logo.png}
    ~\\[2.5cm]
    \begin{center}
    \textbf{\huge Evaluation von Bluetooth, NFC und USB im Rahmen von PharoThings und Android als p2p Verbindung}\\[0.5cm]
    {\Large Bachelorarbeit Sommersemester 2019}
    \vfill
    \end{center}
    ~\\[2.0cm]
    \begin{flushright}
    {\large Jan Phillip Kretzschmar \it{(jan@2denker.de)}}\\[0.1cm]
    ~\\[1.0cm]
    {\large Betreuer (Zweidenker GmbH):}\\[0.1cm]
    {\large  \it{(@2denker.de)}}
    ~\\[0.5cm]
    {\large Betreuer (TH Köln):}\\[0.1cm]
    {\large Prof. Christian Kohls}\\[0.1cm]

	~\\[1.0cm]
    {\large 1. Juli 2019}
	\end{flushright}
    \end{titlepage}
    \pagebreak
	\tableofcontents
	\pagebreak
	
    \section{Expose}
        Die Verbindungskonfiguration von IoT Geräten wurde bereits auf Basis von Wi-Fi Direct untersucht\footnote{Kretzschmar: {\it Verbindungskonfiguration von PharoThings auf Raspberry Pi durch Android App} TH Köln Praxisprojekt Sommersemester (2019).}. Da die Implementierung dieser Technologie einige Unzulänglichkeiten besitzt, soll nun ein Vergleich von weiteren p2p Technologien vorgenommen werden. Hierunter fallen Bluetooth, NFC und USB, da diese alle von Android Smartphones nativ unterstützt werden. Da die Vor- und Nachteile sowie Funktionsweisen dieser p2p Schnittstellen bereits bekannt sind, sollen sie im Rahmen dieser Arbeit in die Verbindungskonfiguration mit PharotThings eingebunden werden, um auch hier Probleme aufzudecken und zu einer optimalen Lösung des Problems zu gelangen.
        
        Die aktuelle Lösung \footnote{https://github.com/janphkre/iot-connectivity} besteht aktuell aus Service Discovery und einer p2p Verbindung. Ob Service Discovery in seiner aktuellen Form beibehalten werden soll, ist ebenfalls zu evaluieren, da sich eine erweiterte Lösung auch nicht auf nur eine Technologie beschränken muss. Es ist daher zu überprüfen, ob eine Hybridlösung in Verbindung mit Wi-Fi Direct oder Bluetooth LE sinnvoll ist, sodass lediglich die p2p Verbindung über eine andere Technologie stattfindet.
        
	\section{Evaluationsziel}        
        Im vorausgegangenen Praxisprojekt wurde deutlich, dass Wi-Fi Direct einige gravierende Mängel im Hinblick auf die Implementierbarkeit und Robustheit der bereitgestellten Bibliotheken aufweist. So ist die Dokumentation der meisten Nachrichten und Events unvollständig oder fehlt und es ist unklar wie eine Verbindung aufgebaut werden muss. Gleichzeitig sind Fehlermeldungen entweder nicht aussagekräftig oder führen dazu, dass der laufende Daemon abstürzt.
        Daraus lassen sich für diese Evaluation zwei Kriterien im Bereich der Robustheit ableiten.
        Zunächst kann getestet werden wie aussagekräftig Fehlercodes oder Fehlermeldungen sind.
        Das eine Ende des Spektrums ist die Darstellung des Erfolgs eines Aufrufs durch eine boolesche Variable, auf der anderen Seite wird ein genauer Fehlercode in Verbindung mit einer Fehlermeldung ausgegeben.
    \section{Evaluationsmethodik}
\end{document}
