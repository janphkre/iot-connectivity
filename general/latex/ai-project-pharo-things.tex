\documentclass[12pt,a4paper]{article}
\usepackage[utf8]{inputenc}
\usepackage{graphicx}
\usepackage[a4paper, textwidth=15cm, textheight=23cm]{geometry}

\title{Praxisprojekt: PharoThings-Connectivity WS18/19}
\begin{document}
	\begin{titlepage}
    \includegraphics[width=0.4\textwidth]{th_logo.png}
    ~\\[2.5cm]
    \begin{center}
    \textbf{\huge Verbindungskonfiguration von PharoThings auf Raspberry Pi durch Android App}\\[0.5cm]
    {\Large Praxisprojekt Wintersemester 2018/2019}
    \vfill
    \end{center}
    ~\\[2.0cm]
    \begin{flushright}
    {\large Jan Phillip Kretzschmar \it{(jan@2denker.de)}}\\[0.1cm]
    ~\\[1.0cm]
    {\large Betreuer (Zweidenker GmbH):}\\[0.1cm]
    {\large Christian Denker \it{(christian@2denker.de)}}
    ~\\[0.5cm]
    {\large Betreuer (TH Köln):}\\[0.1cm]
    {\large ---}\\[0.1cm]

	~\\[1.0cm]
    {\large 19. Oktober 2018}
	\end{flushright}
    \end{titlepage}
    
    \section{Expose}
        Pharo ist eine auf Smalltalk basierende objektorientierte und dynamisch getypte Programmiersprache,
        welche gleichzeitig ihre eigene live Entwicklungsumgebung mit mächtigen Debugging-Tools ist.\footnote{http://pharo.org/}
        PharoThings bietet eine reduzierte Platform für  Internet of Things(IoT), sodass eine Ausführung von Programmen auf Kleinstcomputern möglich ist.
        Um keine Kompromisse im Bereich der Entwicklungsumgebung machen zu müssen, kann über Remote Debugger auf live Programme zugegriffen werden.
        Dadurch ist lediglich die Ausführung auf das IoT Gerät ausgelagert. PharoThings bietet in Verbindung mit WiringPi eine Platform für Raspberry Pi,
        auf der Board Modeling simpel möglich ist.\footnote{https://github.com/pharo-iot/PharoThings}
        Für die Erstkonfiguration und Verbindungskonfigurationvon PharoThings auf Raspberry Pi ist es aktuell nötig,
        diese Konfiguration z.B. der WiFi-Verbindung durch einen Computer vorzunehmen.
        Um die Pharo Things Laufzeitumgebungen als IoT-Geräte simpel nutzen zu können, ist es nötig,
        die Konfiguration der Installationen und Geräte zu vereinfachen. Dabei soll eine Android App diesen Vorgang übernehmen:
        \begin{enumerate}
            \item Erkennen und Auflisten von IoT-Geräten in der Nähe. Die zu verwendende Kommunikationstechnologie ist dabei zu evaluieren.
            \item Verbindungsaufbau zu ausgewähltem IoT-Gerät
            \item IoT-Gerät erhält Hostnamen, WiFi-Konfiguration, Beacon Intervall, etc.
            \item Eventuelle Verbindungsprobleme der erst WiFi-Verbindung werden über die bestehende Verbindung zurückgemeldet
            \item Pharo Things-Installationen im aktuellen WiFi werden aufgelistet.
        \end{enumerate}
        Um diesen Vorgang umsetzen zu können, werden drei Komponenten implementiert:
        \begin{enumerate}
            \item Ein Protokoll muss definiert werden, welches die Kommunikation zu PharoThings Instanzen zur Konfiguration festlegt.
            Weiterhin muss festgelegt werden, in welcher Art und Weise ein Beacon-Signal im WiFi von den Installationen gesendet wird.
            Um das WiFi nicht zu überlasten, empfiehlt es sich diese Nachrichten kurz zu halten. Es ist zu evaluieren,
            ob sich Installationen auch gegenseitig erkennen können, sodass ein gebündelter Beacon gesendet werden kann.
            \item Eine Android App, welche den Nutzer durch den beschriebenen Vorgang leitet, muss implementiert werden.
            Der Fokus hierbei liegt darin, diesen Vorgang mit möglichst wenig Nutzerinteraktion durchzuführen.
            \item Eine Anwendung in PharoThings muss erstellt werden, welche das Protokoll implementiert
            und basierend darauf sich in einem WiFi einwählen kann und ein Beacon-Signal in diesem WiFi sendet.
        \end{enumerate}
        Das Projekt wird mit Unterstützung der Zweidenker GmbH durchgeführt.
    \pagebreak
    \section{Mögliche Kommunikationstechnologien}
        Ein Ad Hoc Netzwerk bietet im Allgemeinen die Möglichkeit {\it peer to peer} (p2p) Verbindungen zwischen Geräten dezentralisiert aufzubauen.
        Geräte können hierbei selbstständig eine Netzwerkverbindung untereinander aushandeln. Da solche Verbindungen nur dann sinnvoll sind,
        wenn es Daten gibt, die nur zwischen den beiden verbundenen Geräten ausgetauscht werden müssen, ergibt ein solches Netzwerk meist nur
        im Bezug auf eine tatsächlliche Anwendung Sinn. Die erweiterte Definition des Ad Hoc Netzwerks
        bezieht somit alle Netzwerkschichten des OSI-Modells mit ein.\footnote{TODO: Axel Sikora - Wireless LAN. Seite 23}
        Obwohl das OSI-Modell vor Allem auf Ethernet und WLAN ausgelegt ist, lässt sich die Definition des Ad Hoc Netzwerks
        dennoch für weitere Kommunikationstechnologie übernehmen, da diese ebenfalls p2p Verbindungen aufbauen können.
        Folgende Punkte müssen erfüllt sein, damit ein Kommunikationsmedium für dieses Projekt genutzt werden kann:
        \begin {enumerate}
        \item {\it Unterstützung in Android Smartphones:}
        Damit eine große Anzahl an potentiellen Nutzern angesprochen werden kann, muss die Verbindungsschnittstelle von Smartphones unterstützt werden.
        Für dieses Projekt wird dabei nur Android betrachtet.
        Aktuelle Smartphones bieten im Allgemeinen zur Zeit die vier Schnittstellen {\bf USB, NFC, Bluetooth und WiFi},
        über die sich Verbindungen zu Geräten in der näheren Umgebung aufbauen lassen.
        \item {\it Hardware an IoT Geräten:}
        Als IoT Gerät dient in diesem Projekt ein Raspberry Pi.
        In den Varianten {\it Model 3 B, Model 3 B+ and Model Zero W} bietet Dieser USB, Bluetooth und WiFi als mögliche Schnittstellen.
        Durch die Verwendung der GPIO-Pins ist es außerdem möglich, ein NFC-Modul anzubinden,
        jedoch würde Dies die später nutzbaren Pins unerwünscht einschränken.
        Weiterhin bietet der Raspberry Pi ein vollständiges Betriebssystem mit Benutzeroberfläche, jedoch soll eine Internetverbindung
        ohne Peripherie am Raspberry Pi konfiguriert werden können. 
        \item {\it ???}
        \end {enumerate}
        \paragraph{Kommunikation über WLAN}
            Um eine Kommunikation im Rahmen der WLAN-Definition zu ermöglichen, ist es nötig, einen Verbindungsaufbau in der {\bf Data Link Layer}
            (Layer 2 des OSI-Modells) anzusiedeln, da sich beide Geräte nicht in einem Netzwerk befinden, in dem das Internet Protocol möglich ist.
            Außerdem können so bestehende WLAN-Verbindungen beibehalten werden.
            Da die Data Link Layer die Gerätetreiber-Ebene wiederspiegelt, ist es nötig, das {\bf Media Access Control}-Protocol hierbei zu nutzen.
            Dieses Protokoll regelt den Zugriff auf das Übertragungsmedium durch unterschiedliche Wartezeiten zwischen Paketen und die Reservierung des Mediums zum Senden von Paketen.
            Da das MAC-Protokoll zudem die Addressierung von Geräten ermöglicht, bietet es bereits die Möglichkeit, Broadcasts zu senden.
            Um zudem die hohe Fehleranfälligkeit eines Drahtlosnetzwerkes für höhere Schichten zu reduzieren,
            wird jedes Paket vom Empfänger bestätigt.\footnote{TODO: Martin Sauter - Grundkurs Mobile Kommunikationssysteme 6.Auflage Seite 325ff.}

            Um dieses Protokoll in Linux verwenden zu können, ist es nötig, Sockets mit dem Attribut {\bf SOCK\_RAW} zu öffnen,
            um eigene MAC-Pakete senden zu können. Solche Sockets können jedoch nur mit der Berechtigung {\bf CAP\_NET\_RAW} erstellt werden.\footnote{http://man7.org/linux/man-pages/man7/packet.7.html}
            Unter Android fällt diese Berechtigung mangels Granularität root zu, wodurch diese Lösung unpraktikabel wird,
            wenn eine möglichst große Nutzergruppe angesprochen werden soll.\footnote{https://elinux.org/Android\_Security\#Paranoid\_network-ing}
            Android bietet jedoch ab API 14 die Möglichkeit, peer-to-peer Verbindungen über {\bf WiFi-Direct} aufzubauen. ... WIFI DIRECT ERKLÄREN
            Hierzu ist es nötig in Android eingebaute WiFiManager zu nutzen, ein kurzer Test hat dabei ergeben, dass bestehende WiFi-Verbindungen abgebrochen werden.\footnote{TODO: Github repository}
        \paragraph{Exkurs: AirPlay}
            
        \paragraph{Kommunikation über Bluetooth}

        \paragraph{Kommunikation über NFC}

        \paragraph{Kommunikation über USB}
\end{document}
