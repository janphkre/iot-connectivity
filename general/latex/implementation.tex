\section{Umsetzung}
Um eine stabile Lösung schnell und simpel anbieten zu können, wird zur Implementierung der REST-Schnittstelle auf bestehende Server sowie Client Bibliotheken zurückgegriffen. Diese bieten somit die Möglichkeit die Implementierung dieser Schnittstelle deskriptiv vorzunehmen. In Android wird somit die Implementierung zur Laufzeit aus einer Schnittstellenbeschreibung generiert und auf der Serverseite müssen so nur die Implementierungen der Methoden beschrieben werden. Da die Wahl des Übertragungsmediums auf WiFi-Direct gefallen ist, können ebenso Teile des Betriebssystems und Standardbibliotheken zum Broadcast und Aufbau der p2p Verbindung genutzt werden.
Da unter Linux das Paket wpa\_supplicant\footnote{https://w1.fi/wpa\_supplicant/} auf allen gängigen Distributionen, so auch unter raspbian auf dem Raspberry Pi, dazu genutzt wird, WiFi Schnittstellen und Netzwerke zu verwalten. Es läuft als daemon im Betriebssystem mit und kann über ein Kommandozeileninterface bedient werden oder mit einer C-Schnittstelle in das eigene Programm eingebunden werden.
\subsection{Pharo}
\subsection{Android}

\subsection{Stolpersteine}
Theoretisch blockieren sich die Verwendung von WiFi-Direct und gewöhnlichem WiFi mit einem Access Point auf der selben Netzwerkschnittstelle, da diese eine Sende- und Empfangseinheit exklusiv benötigen. Diese Limitierung wird bei Nutzung des wpa\_supplicant jedoch nicht deutlich, da gewöhnliche Schnittstelle und p2p Schnittstelle immer als separate Einheiten auf der gleichen Netzwerkkarte aufgelistet werden. Da manche Netzwerkkarten inzwischen mehr als ein Radio enthalten, um höhere Übertragungsraten zu ermöglichen, kann dies jedoch auf manchen Geräten dennoch funktionieren. Im Falle des Raspberry Pi 3 B ist dies jedoch aktuell nicht möglich und eine zweite Netzwerkkarte wird benötigt, um WiFi Direct und WiFi parallel zu nutzen. Diese Limitierung gilt jedoch nicht für die Service Discovery, die über WiFi Direct stattfindet.