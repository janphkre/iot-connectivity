\section{Ausblick}
	Im Rahmen dieses Projektes wurde Wi-Fi Direct zur Nutzung als p2p Technologie zwischen Android und pharo untersucht. Zwecks mangelnder Dokumentation von Wi-Fi Direct und dessen Implementierung unter Linux war es nicht möglich, eine stabile Verbindung wiederholbar aufzubauen. Da Service Discovery jedoch größtenteils stabil über Wi-Fi Direct funktioniert, ist zu überlegen, ob eine Hybrid Lösung in Verbindung mit Bluetooth oder anderen p2p Technologien sinnvoll ist. Damit wäre es möglich, manche Nachteile von anderen Technologien auszugleichen. In Kombination mit Bluetooth könnte so die Broadcastreichweite spürbar erhöht werden. Generell sollten somit andere Technologien auch auf deren Nutzbarkeit überprüft werden. Für Diese wird es dann auch nötig, einen Wrapper um Socketverbindungen zu bauen.
	Ebenso ist es denkbar, mehrere p2p Technologien parallel zu betreiben, damit der Nutzer nicht durch eine fehlende Technologie in seinem Smartphone eingeschränkt wird. Hierbei sollte auch darauf geachtet werden, dass der Nutzer nicht von dieser Parallelität behelligt wird, somit die Auswahl automatisch geschieht.
	
	Der implementierte REST-Server verwendet noch den standardmäßigen Error Handler, der im Fehlerfall einen Stacktrace als plain text zurückliefert. Dies sollte abgefangen werden und in ein JSON Objekt mit Fehlercodes und sprechenden Fehlermeldungen umgewandelt werden. Aktuell führt der REST-Server noch alle Aufrufe ohne Authentifizierung durch. Es sollte jedoch wenigstens eine Authentifizierung bei der Erstkonfiguration festgelegt werden, sodass nicht autorisierte Personen keine Änderungen an den Einstellungen vornehmen können.
	In Verbindung mit dieser Authentifizierung auf Anwendungsebene kann auch das Gerät an ein virtuelles Netzwerk gebunden werden, welches eigene Geräte bündelt. Dieses virtuelle Netzwerk wird dazu genutzt, dass die Geräte einen Server in diesem Netzwerk besitzen, auf den sie sich beziehen können. Dies kann entweder eine lokal laufende pharo Instanz sein oder ein Cloudservice, an dem das virtuelle Netzwerk einem Account entspricht. Damit wird nicht nur der Zugriff lokal sondern auch Remote eingeschränkt. Um diese virtuellen Netzwerke gut verwalten zu können, empfiehlt es sich auch hierbei ein DNS Service Discovery einzusetzen, die Daten jedoch durch verschiedene Maßnahmen zu schützen.\cite[S.8]{AI-Kaiser2}
	
	Im Rahmen von Low Energy Netzwerken könnten die Broadcasts der Service Discovery von anderen pharo Instanzen auch erkannt werden und so gebündelt werden. Dies erlaubt es, Energie dadurch einzusparen, dass nur noch ein Gerät aus dem Low Energy Netzwerk den Broadcastanfragen antworten muss. Ebenfalls lässt sich so die Reichweite von p2p Verbindungen erhöhen, da die Anfragen zum entsprechenden Ziel weitergeleitet werden können. Im Hinblick auf Bandbreite stellt dies erst ein Problem dar, wenn mehrere Geräte im selben Netzwerk parallel administriert werden, was jedoch einen Randfall darstellen sollte und somit vorerst ignoriert werden kann. Es ist somit also denkbar die pharo Instanzen als Knoten in einem Netzwerk aus oben genannten Gründen agieren zu lassen. Diese Knoten müssten dann jedoch ebenfalls Routingtabellen verwalten, was dieses dezentrales Netzwerk im Vergleich zum Nutzen, der daraus gewonnen wird, zu kompliziert macht.
	
	
